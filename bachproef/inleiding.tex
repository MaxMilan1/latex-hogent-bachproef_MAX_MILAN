%%=============================================================================
%% Inleiding
%%=============================================================================

\chapter{\IfLanguageName{dutch}{Inleiding}{Introduction}}%
\label{ch:inleiding}

De inleiding moet de lezer net genoeg informatie verschaffen om het onderwerp te begrijpen en in te zien waarom de onderzoeksvraag 
de moeite waard is om te onderzoeken. In de inleiding ga je literatuurverwijzingen beperken, zodat de tekst vlot leesbaar blijft. 
Je kan de inleiding verder onderverdelen in secties als dit de tekst verduidelijkt. 
Zaken die aan bod kunnen komen in de inleiding~\autocite{Pollefliet2011}:

\begin{itemize}
  \item context, achtergrond
  \item afbakenen van het onderwerp
  \item verantwoording van het onderwerp, methodologie
  \item probleemstelling
  \item onderzoeksdoelstelling
  \item onderzoeksvraag
  \item \ldots
\end{itemize}

\section{\IfLanguageName{dutch}{Probleemstelling}{Problem Statement}}%
\label{sec:probleemstelling}

Projecten worden vaak niet goed gedocumenteerd, dit kan leiden tot problemen in de toekomst. Wanneer een andere persoon de code van een ongedocumenteerd project wilt gebruiken moet de code volledig gelezen worden voordat er begrepen wordt wat de code doet. 
Dit is een tijdrovend proces en kan voorkomen worden door goede documentatie.
Wanneer de code jaren later aangepast moet worden is het ook handig om goede documentatie te hebben, zodat de persoon weet waar er aanpassingen moeten gebeuren.
De skills en know-how van een project kunnen verloren gaan wanneer er geen documentatie is.
Deze dienen juist gedeeld te worden met anderen zodat er geen dubbel werk gedaan moet worden.
Het is dus belangrijk dat er aan documentatie gedaan wordt en dat deze up-to-date blijft.

Het documenteren van een project is iets wat veel tijd kost en wat meestal geen aandacht krijgt.
Een tool die dit proces kan versnellen / automatiseren zou een grote meerwaarde zijn.
De tool bestaat uit een geautomatiseerde documentatie LLM die de project code analyseert en samenvat in een document. 
Dit geeft de lezers de mogelijkheid om zich in te lezen in het project en erna zelf aanpassingen te maken of stukken code te gebruiken voor een ander project.

\section{\IfLanguageName{dutch}{Onderzoeksvraag}{Research question}}%
\label{sec:onderzoeksvraag}

Hoe kan geautomatiseerde documentatiegeneratie met behulp van Large Language Modellen (LLM) effectief worden toegepast om duidelijke en informatieve overzichten te produceren voor Python projecten?

\section{\IfLanguageName{dutch}{Onderzoeksdoelstelling}{Research objective}}%
\label{sec:onderzoeksdoelstelling}

Het eindresultaat van deze bachelorproef is een Proof of Concept (PoC) van een geautomatiseerde tool die de project code analyseert en er documentatie van genereert.
De gegenereerde documentatie laat het toe het project te begrijpen zonder er te veel tijd aan te besteden.

\section{\IfLanguageName{dutch}{Opzet van deze bachelorproef}{Structure of this bachelor thesis}}%
\label{sec:opzet-bachelorproef}

% Het is gebruikelijk aan het einde van de inleiding een overzicht te
% geven van de opbouw van de rest van de tekst. Deze sectie bevat al een aanzet
% die je kan aanvullen/aanpassen in functie van je eigen tekst.

De rest van deze bachelorproef is als volgt opgebouwd:

In Hoofdstuk~\ref{ch:stand-van-zaken} wordt een overzicht gegeven van de stand van zaken binnen het onderzoeksdomein, op basis van een literatuurstudie.

In Hoofdstuk~\ref{ch:methodologie} wordt de methodologie toegelicht en worden de gebruikte onderzoekstechnieken besproken om een antwoord te kunnen formuleren op de onderzoeksvragen.

% TODO: Vul hier aan voor je eigen hoofstukken, één of twee zinnen per hoofdstuk

In Hoofdstuk~\ref{ch:conclusie}, tenslotte, wordt de conclusie gegeven en een antwoord geformuleerd op de onderzoeksvragen. Daarbij wordt ook een aanzet gegeven voor toekomstig onderzoek binnen dit domein.