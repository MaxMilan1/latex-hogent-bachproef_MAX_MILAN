%%=============================================================================
%% Inleiding
%%=============================================================================

\chapter{\IfLanguageName{dutch}{Inleiding}{Introduction}}%
\label{ch:inleiding}

% TODOOOOO
% Inleiding schrijven
In de wereld van softwareontwikkeling is documentatie een belangrijk onderdeel van een project.
Documentatie is een manier om de code te beschrijven en te verklaren wat de code doet.
Het is een manier om de code te begrijpen zonder dat de code zelf gelezen moet worden.
Documentatie is belangrijk voor het onderhouden van een project, het delen van kennis en het begrijpen van de code.

Bestaande tools vergen gedocumenteerde code om de documentatie in andere formaten te genereren.
Dit zorgt ervoor dat de code manueel gedocumenteerd moet worden.
Een tool die dit automatisch kan doen zorgt ervoor dat dit geen manuele taak meer is.
Dit vergemakkelijkt het proces van het documenteren van een project en helpt met het voorkomen van fouten die erin kunnen sluipen. 
In dit onderzoek wordt er gekeken naar hoe een enkel bestand gedocumenteerd kan worden en vervolgens hoe verschillende bestanden in een project samen gedocumenteerd kunnen worden om zo een overzicht van het project te geven.
Zodat de lezer van de documentatie een goed beeld krijgt van het project en er zelf mee aan de slag kan gaan.

\section{\IfLanguageName{dutch}{Probleemstelling}{Problem Statement}}%
\label{sec:probleemstelling}

Projecten worden vaak niet goed gedocumenteerd, wat kan leiden tot problemen in de toekomst. Wanneer een andere persoon de code van een ongedocumenteerd project wil gebruiken moet de code volledig gelezen worden voordat er begrepen wordt wat de code doet. 
Dit is een tijdrovend proces en kan voorkomen worden door goede documentatie.
Wanneer de code jaren later aangepast moet worden is het ook handig om goede documentatie te hebben, zodat de persoon weet waar er aanpassingen moeten gebeuren.
De skills en know-how van een project kunnen verloren gaan wanneer er geen documentatie is.
Deze dienen juist gedeeld te worden met anderen zodat er geen dubbel werk gedaan moet worden.
Het is dus belangrijk dat er aan documentatie gedaan wordt en dat deze up-to-date blijft.

Er wordt gekeken naar een tool die automatisch de documentatie van een project kan genereren.
Verder wordt er uitgelegd waarom een Large Language Model (LLM) gebruikt kan worden om de documentatie te genereren.
Dit geeft de lezers de mogelijkheid om zich in te lezen in het project en erna zelf aanpassingen te maken of stukken code te gebruiken voor een ander project.

Python is een populaire programmeertaal die gebruikt wordt in de IT wereld.
Volgens \textcite{TIOBE2024} een website die zoekpagina's afgaat en de populariteit van programmeertalen op basis van het aantal hits bepaalt, staat Python op de eerste plaats.
Het is dus interessant om een tool te maken die Pythonprojecten kan documenteren.

\section{\IfLanguageName{dutch}{Onderzoeksvraag}{Research question}}%
\label{sec:onderzoeksvraag}

Hoe kan geautomatiseerde documentatiegeneratie met behulp van een Large Language Modellen (LLM) effectief worden toegepast op ongedocumenteerde Pythonprojecten om er duidelijke en overzichtelijke documentatie van te maken? 

\begin{itemize}
  \item Wat is documentatie?
  \item Wat zijn de huidige documentatie tools?
  \item Wat is er nodig om de code van een bestand te documenteren?
  \item Wat is er nodig om de code van een project te documenteren?
  \item Waarom documentatie met behulp van een LLM?
  \item Hoe wordt de documentatie zo goedkoop mogelijk gehouden?
\end{itemize}

\section{\IfLanguageName{dutch}{Onderzoeksdoelstelling}{Research objective}}%
\label{sec:onderzoeksdoelstelling}

Het eindresultaat van deze bachelorproef is een Proof of Concept (PoC) van een geautomatiseerde tool die de code van een Pythonproject analyseert en er documentatie van genereert.
De gegenereerde documentatie laat het toe het project te begrijpen.

\section{\IfLanguageName{dutch}{Opzet van deze bachelorproef}{Structure of this bachelor thesis}}%
\label{sec:opzet-bachelorproef}

% Het is gebruikelijk aan het einde van de inleiding een overzicht te
% geven van de opbouw van de rest van de tekst. Deze sectie bevat al een aanzet
% die je kan aanvullen/aanpassen in functie van je eigen tekst.

De rest van deze bachelorproef is als volgt opgebouwd:

In Hoofdstuk~\ref{ch:stand-van-zaken} wordt een overzicht gegeven van de stand van zaken binnen het onderzoeksdomein, op basis van een literatuurstudie.

In Hoofdstuk~\ref{ch:methodologie} wordt de methodologie toegelicht en worden de gebruikte onderzoekstechnieken besproken om een antwoord te kunnen formuleren op de onderzoeksvragen.

% TODO: Vul hier aan voor je eigen hoofstukken, één of twee zinnen per hoofdstuk

In Hoofdstuk~\ref{ch:resultaten} worden de resultaten van het onderzoek besproken, dit voor bestanddocumentatie en projectdocumentatie.
In het verdere verloop van dit onderzoek wordt er gekeken naar de evaluatie van de gegenereerde documentatie.

In Hoofdstuk~\ref{ch:conclusie}, tenslotte, wordt de conclusie gegeven en een antwoord geformuleerd op de onderzoeksvragen. Daarbij wordt ook een aanzet gegeven voor toekomstig onderzoek binnen dit domein.