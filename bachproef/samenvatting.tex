%%=============================================================================
%% Samenvatting
%%=============================================================================

% TODO: De "abstract" of samenvatting is een kernachtige (~ 1 blz. voor een
% thesis) synthese van het document.
%
% Een goede abstract biedt een kernachtig antwoord op volgende vragen:
%
% 1. Waarover gaat de bachelorproef?
% 2. Waarom heb je er over geschreven?
% 3. Hoe heb je het onderzoek uitgevoerd?
% 4. Wat waren de resultaten? Wat blijkt uit je onderzoek?
% 5. Wat betekenen je resultaten? Wat is de relevantie voor het werkveld?
%
% Daarom bestaat een abstract uit volgende componenten:
%
% - inleiding + kaderen thema
% - probleemstelling
% - (centrale) onderzoeksvraag
% - onderzoeksdoelstelling
% - methodologie
% - resultaten (beperk tot de belangrijkste, relevant voor de onderzoeksvraag)
% - conclusies, aanbevelingen, beperkingen
%
% LET OP! Een samenvatting is GEEN voorwoord!

%%---------- Samenvatting -----------------------------------------------------
% De samenvatting in de hoofdtaal van het document

\chapter*{\IfLanguageName{dutch}{Samenvatting}{Abstract}}

Deze bachelorproef richt zich op het documenteren van ongedocumenteerde Pythonprojecten met behulp van een Large Language Model (LLM).
Het genereren van duidelijk en overzichtelijke documentatie van een project helpt bij het begrijpen van de code en is de eerste stap in het delen van kennis.
Bestaande tools hebben echter gedocumenteerde code nodig om documentatie te genereren. 
Het automatiseren van het documentatieproces zorgt ervoor dat dit geen manuele taak meer is.  

De centrale onderzoeksvraag is: "Hoe kan geautomatiseerde documentatiegeneratie met behulp van Large Language Modellen (LLM) effectief worden toegepast op ongedocumenteerde Pythonprojecten om er duidelijke en overzichtelijke documentatie van te maken?" 
Deze vraag is verder opgesplitst in enkele deelvragen, wat is documentatie en wat is er nodig om een bestand en een project te documenteren?
Met als doel een Proof of Concept (PoC) van een geautomatiseerde tool die de code van een Pythonproject analyseert en er documentatie van genereert.

Het onderzoek omvat enkele fases. Eerst wordt er gekeken hoe een enkel bestand gedocumenteerd kan worden. 
Vervolgens wordt er gekeken hoe verschillende bestanden in een project samen gedocumenteerd kunnen worden om zo een overzicht van het project te geven.
De laatste fase beslaagd het evalueren van de tool door de documentatie van de tool te vergelijken met de handgeschreven documentatie van een project.

De resultaten van de evaluatie tonen aan dat de documentatie van de tool en de handgeschreven documentatie gelijkaardig zijn en door de visuele weergave van de relaties tussen de bestanden is de documentatie overzichtelijk en duidelijk.
Er kunnen echter wel enkele fouten in de documentatie sluipen.

Deze studie biedt een oplossing voor het documenteren van ongedocumenteerde Pythonprojecten en kan gebruikt worden om de kennis van een project te delen met anderen.
