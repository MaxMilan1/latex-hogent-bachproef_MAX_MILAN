%===============================================================================
% LaTeX sjabloon voor de bachelorproef toegepaste informatica aan HOGENT
% Meer info op https://github.com/HoGentTIN/latex-hogent-report
%===============================================================================

\documentclass[dutch,dit,thesis]{hogentreport}

% \usepackage{lipsum} % For blind text, can be removed after adding actual content

%% Pictures to include in the text can be put in the graphics/ folder
\graphicspath{{graphics/}}

%% For source code highlighting, requires pygments to be installed
%% Compile with the -shell-escape flag!
%% \usepackage[chapter]{minted}
%% If you compile with the make_thesis.{bat,sh} script, use the following
%% import instead:
%%\usepackage[chapter,outputdir=../output]{minted}
\usepackage[chapter]{minted}
\usemintedstyle{solarized-light}

%% Formatting for minted environments.
\setminted{%
    autogobble,
    frame=lines,
    breaklines,
    linenos,
    tabsize=4
}

%% Ensure the list of listings is in the table of contents
\renewcommand\listoflistingscaption{%
    \IfLanguageName{dutch}{Lijst van codefragmenten}{List of listings}
}
\renewcommand\listingscaption{%
    \IfLanguageName{dutch}{Codefragment}{Listing}
}
\renewcommand*\listoflistings{%
    \addcontentsline{toc}{chapter}{\listoflistingscaption}%
    \listof{listing}{\listoflistingscaption}%
}

%% For source code highlighting, requires pygments to be installed
%% Compile with the -shell-escape flag!
%%\usepackage[chapter]{minted}
\usepackage{caption}
\usepackage{subcaption}

%% Change this line to edit the line numbering style:
\renewcommand{\theFancyVerbLine}{\ttfamily\scriptsize\arabic{FancyVerbLine}}

%% Macro definition to load external python source files with \pythoncode{filename}:
% \newmintedfile[pythoncode]{python}{
%     bgcolor=bg,
%     fontfamily=tt,
%     linenos=true,
%     numberblanklines=true,
%     numbersep=5pt,
%     gobble=0,
%     framesep=2mm,
%     funcnamehighlighting=true,
%     tabsize=4,
%     obeytabs=false,
%     breaklines=true,
%     mathescape=false
%     samepage=false,
%     showspaces=false,
%     showtabs =false,
%     texcl=false,
% }

% Other packages not already included can be imported here

%%---------- Document metadata -------------------------------------------------
% TODO: Replace this with your own information
\author{Max Milan}
\supervisor{Dhr. G. Bosteels}
\cosupervisor{Dhr. A. Pannemans}
\title%[Optionele ondertitel]%
    {Geautomatiseerde documentatie generatie met behulp van Large Language Modellen: Het genereren van duidelijke overzichten en informatieve beschrijvingen voor ongedocumenteerde Python projecten}
\academicyear{\advance\year by -1 \the\year--\advance\year by 1 \the\year}
\examperiod{1}
\degreesought{\IfLanguageName{dutch}{Professionele bachelor in de toegepaste informatica}{Bachelor of applied computer science}}
\partialthesis{false} %% To display 'in partial fulfilment'
%\institution{Internshipcompany BVBA.}

%% Add global exceptions to the hyphenation here
\hyphenation{back-slash}

%% The bibliography (style and settings are  found in hogentthesis.cls)
\addbibresource{bachproef.bib}            %% Bibliography file
\addbibresource{../voorstel/voorstel.bib} %% Bibliography research proposal
\defbibheading{bibempty}{}

%% Prevent empty pages for right-handed chapter starts in twoside mode
\renewcommand{\cleardoublepage}{\clearpage}

\renewcommand{\arraystretch}{1.2}

%% Content starts here.
\begin{document}

%---------- Front matter -------------------------------------------------------

\frontmatter

\hypersetup{pageanchor=false} %% Disable page numbering references
%% Render a Dutch outer title page if the main language is English
\IfLanguageName{english}{%
    %% If necessary, information can be changed here
    \degreesought{Professionele Bachelor toegepaste informatica}%
    \begin{otherlanguage}{dutch}%
       \maketitle%
    \end{otherlanguage}%
}{}

%% Generates title page content
\maketitle
\hypersetup{pageanchor=true}

%%=============================================================================
%% Voorwoord
%%=============================================================================

\chapter*{\IfLanguageName{dutch}{Woord vooraf}{Preface}}%
\label{ch:voorwoord}

%% TODO:
%% Het voorwoord is het enige deel van de bachelorproef waar je vanuit je
%% eigen standpunt (``ik-vorm'') mag schrijven. Je kan hier bv. motiveren
%% waarom jij het onderwerp wil bespreken.
%% Vergeet ook niet te bedanken wie je geholpen/gesteund/... heeft

Toen ik aan de richting Toegepaste Informatica begon, wist ik meteen welke specialisatie ik wilde volgen. 
De keuze richting AI en Data Engineering sprak mij direct aan. 
Deze richting biedt mij de mogelijkheid aan om steeds nieuwe uitdagingen te vinden in alles wat ik doe.

Het aangaan van een uitdaging geeft mij de motivatie om steeds het beste van mezelf te geven. 
Zo was het ook een uitdaging om deze bachelorproef tot een goed einde te brengen. 
Een van de grootste uitdagingen was het werken met LLM's, aangezien dit de eerste keer was dat ik deze zelf implementeerde.

Daarom wil ik graag mijn co-promotor, Arne Pannemans, bedanken voor zijn onophoudelijke steun en waardevolle bijdragen. 
Zijn voortdurende aanwezigheid bij het geven van feedback en zijn kritische blik waren van onschatbare waarde voor het tot stand brengen van deze bachelorproef. 
Zijn continue steun en toewijding hebben mij geïnspireerd en gemotiveerd om het beste uit mezelf te halen. 
Dankzij zijn begeleiding heb ik mijn onderzoek naar een hoger niveau kunnen tillen.

Ook wil ik mijn promotoren meneer Gert-Jan Bosteels en mevrouw Lena De Mol bedanken voor hun kritische en waardevolle feedback. 
Deze feedback heeft ervoor gezorgd dat ik de juiste richting uitging en dat ik mijn ideeën helder kon verwoorden.

Tot slot zou ik graag mijn ouders en vrienden willen bedanken. Hun steun, voortdurende aanmoediging en begrip hebben mij door alle moeilijke momenten geholpen. 
Dankzij hen heb ik deze bachelorproef tot een goed einde kunnen brengen.

%%=============================================================================
%% Samenvatting
%%=============================================================================

% TODO: De "abstract" of samenvatting is een kernachtige (~ 1 blz. voor een
% thesis) synthese van het document.
%
% Een goede abstract biedt een kernachtig antwoord op volgende vragen:
%
% 1. Waarover gaat de bachelorproef?
% 2. Waarom heb je er over geschreven?
% 3. Hoe heb je het onderzoek uitgevoerd?
% 4. Wat waren de resultaten? Wat blijkt uit je onderzoek?
% 5. Wat betekenen je resultaten? Wat is de relevantie voor het werkveld?
%
% Daarom bestaat een abstract uit volgende componenten:
%
% - inleiding + kaderen thema
% - probleemstelling
% - (centrale) onderzoeksvraag
% - onderzoeksdoelstelling
% - methodologie
% - resultaten (beperk tot de belangrijkste, relevant voor de onderzoeksvraag)
% - conclusies, aanbevelingen, beperkingen
%
% LET OP! Een samenvatting is GEEN voorwoord!

%%---------- Samenvatting -----------------------------------------------------
% De samenvatting in de hoofdtaal van het document

\chapter*{\IfLanguageName{dutch}{Samenvatting}{Abstract}}

Deze bachelorproef richt zich op het documenteren van ongedocumenteerde Pythonprojecten met behulp van een Large Language Model (LLM).
Het genereren van duidelijk en overzichtelijke documentatie van een project helpt bij het begrijpen van de code en is de eerste stap in het delen van kennis.
Bestaande tools hebben echter gedocumenteerde code nodig om documentatie te genereren. 
Het automatiseren van het documentatieproces zorgt ervoor dat dit geen manuele taak meer is.  

De centrale onderzoeksvraag is: "Hoe kan geautomatiseerde documentatiegeneratie met behulp van Large Language Modellen (LLM) effectief worden toegepast op ongedocumenteerde Pythonprojecten om er duidelijke en overzichtelijke documentatie van te maken?" 
Deze vraag is verder opgesplitst in enkele deelvragen, wat is documentatie en wat is er nodig om een bestand en een project te documenteren?
Met als doel een Proof of Concept (PoC) van een geautomatiseerde tool die de code van een Pythonproject analyseert en er documentatie van genereert.

Het onderzoek omvat enkele fases. Eerst wordt er gekeken hoe een enkel bestand gedocumenteerd kan worden. 
Vervolgens wordt er gekeken hoe verschillende bestanden in een project samen gedocumenteerd kunnen worden om zo een overzicht van het project te geven.
De laatste fase beslaagd het evalueren van de tool door de documentatie van de tool te vergelijken met de handgeschreven documentatie van een project.

De resultaten van de evaluatie tonen aan dat de documentatie van de tool en de handgeschreven documentatie gelijkaardig zijn en door de visuele weergave van de relaties tussen de bestanden is de documentatie overzichtelijk en duidelijk.
Er kunnen echter wel enkele fouten in de documentatie sluipen.

Deze studie biedt een oplossing voor het documenteren van ongedocumenteerde Pythonprojecten en kan gebruikt worden om de kennis van een project te delen met anderen.


%---------- Inhoud, lijst figuren, ... -----------------------------------------

\tableofcontents

% In a list of figures, the complete caption will be included. To prevent this,
% ALWAYS add a short description in the caption!
%
%  \caption[short description]{elaborate description}
%
% If you do, only the short description will be used in the list of figures

\listoffigures
\listoftables
\listoflistings

% If you included tables and/or source code listings, uncomment the appropriate
% lines.

% Als je een lijst van afkortingen of termen wil toevoegen, dan hoort die
% hier thuis. Gebruik bijvoorbeeld de ``glossaries'' package.
% https://www.overleaf.com/learn/latex/Glossaries

%---------- Kern ---------------------------------------------------------------

\mainmatter{}

% De eerste hoofdstukken van een bachelorproef zijn meestal een inleiding op
% het onderwerp, literatuurstudie en verantwoording methodologie.
% Aarzel niet om een meer beschrijvende titel aan deze hoofdstukken te geven of
% om bijvoorbeeld de inleiding en/of stand van zaken over meerdere hoofdstukken
% te verspreiden!

%%=============================================================================
%% Inleiding
%%=============================================================================

\chapter{\IfLanguageName{dutch}{Inleiding}{Introduction}}%
\label{ch:inleiding}

% TODOOOOO
% Inleiding schrijven
In de wereld van softwareontwikkeling is documentatie een belangrijk onderdeel van een project.
Documentatie is een manier om de code te beschrijven en te verklaren wat de code doet.
Het is een manier om de code te begrijpen zonder dat de code zelf gelezen moet worden.
Documentatie is belangrijk voor het onderhouden van een project, het delen van kennis en het begrijpen van de code.

Bestaande tools vergen gedocumenteerde code om de documentatie in andere formaten te genereren.
Dit zorgt ervoor dat de code manueel gedocumenteerd moet worden.
Een tool die dit automatisch kan doen zorgt ervoor dat dit geen manuele taak meer is.
Dit vergemakkelijkt het proces van het documenteren van een project en helpt met het vermijden van fouten die erin kunnen sluipen. 
In dit onderzoek wordt er gekeken hoe een enkel bestand gedocumenteerd kan worden en vervolgens hoe verschillende bestanden in een project samen gedocumenteerd kunnen worden om zo een overzicht van het project te geven.
Zo krijgt de lezer een goed beeld van de documentatie van het project en kan het project gebruikt worden.

\section{\IfLanguageName{dutch}{Probleemstelling}{Problem Statement}}%
\label{sec:probleemstelling}

Projecten worden vaak niet goed gedocumenteerd, wat kan leiden tot problemen in de toekomst. Wanneer een andere persoon de code van een ongedocumenteerd project wil gebruiken moet de code volledig gelezen worden voordat er begrepen wordt wat de code doet. 
Dit is een tijdrovend proces en kan voorkomen worden door goede documentatie.
Wanneer de code jaren later aangepast moet worden, is het ook handig om goede documentatie te hebben, zodat de persoon weet waar er aanpassingen moeten gebeuren.
De skills en know-how van een project kunnen verloren gaan wanneer er geen documentatie is.
Deze dienen juist gedeeld te worden met anderen zodat er geen dubbel werk gedaan moet worden.
Het is dus belangrijk dat er aan documentatie gedaan wordt en dat deze up-to-date blijft.

Er wordt gekeken naar een tool die automatisch de documentatie van een project kan genereren.
Verder wordt er uitgelegd waarom een Large Language Model (LLM) gebruikt kan worden om de documentatie te genereren.
Dit geeft de lezers de mogelijkheid om zich in te lezen in het project en erna zelf aanpassingen te maken of stukken code te gebruiken voor een ander project.

Python is een populaire programmeertaal die gebruikt wordt in de IT wereld.
Volgens \textcite{TIOBE2024}, een website die zoekpagina's afgaat en de populariteit van programmeertalen op basis van het aantal hits bepaalt, staat Python op de eerste plaats.
Het is dus interessant om een tool te maken die Pythonprojecten kan documenteren.

\section{\IfLanguageName{dutch}{Onderzoeksvraag}{Research question}}%
\label{sec:onderzoeksvraag}

Hoe kan geautomatiseerde documentatiegeneratie met behulp van een Large Language Modellen (LLM) effectief worden toegepast op ongedocumenteerde Pythonprojecten om er duidelijke en overzichtelijke documentatie van te maken? 

\begin{itemize}
  \item Wat is documentatie?
  \item Wat zijn de huidige documentatie tools?
  \item Wat is er nodig om de code van een bestand te documenteren?
  \item Wat is er nodig om de code van een project te documenteren?
  \item Waarom documentatie met behulp van een LLM?
  \item Hoe wordt de documentatie zo goedkoop mogelijk gehouden?
\end{itemize}

\section{\IfLanguageName{dutch}{Onderzoeksdoelstelling}{Research objective}}%
\label{sec:onderzoeksdoelstelling}

Het eindresultaat van deze bachelorproef is een Proof of Concept (PoC) van een geautomatiseerde tool die de code van een Pythonproject analyseert en er documentatie van genereert.
De gegenereerde documentatie laat het toe het project te begrijpen.

\section{\IfLanguageName{dutch}{Opzet van deze bachelorproef}{Structure of this bachelor thesis}}%
\label{sec:opzet-bachelorproef}

% Het is gebruikelijk aan het einde van de inleiding een overzicht te
% geven van de opbouw van de rest van de tekst. Deze sectie bevat al een aanzet
% die je kan aanvullen/aanpassen in functie van je eigen tekst.

De volgende hoofdstukken van deze bachelorproef is als volgt opgebouwd:

In Hoofdstuk~\ref{ch:stand-van-zaken} wordt een overzicht gegeven van de stand van zaken binnen het onderzoeksdomein op basis van een literatuurstudie.

In Hoofdstuk~\ref{ch:methodologie} wordt de methodologie toegelicht en worden de gebruikte onderzoekstechnieken besproken om een antwoord te kunnen formuleren op de onderzoeksvragen.

% TODO: Vul hier aan voor je eigen hoofstukken, één of twee zinnen per hoofdstuk

In Hoofdstuk~\ref{ch:resultaten} worden de resultaten van het onderzoek besproken voor bestanddocumentatie en projectdocumentatie.
In het verdere verloop van dit onderzoek wordt er gekeken naar de evaluatie van de gegenereerde documentatie.

In Hoofdstuk~\ref{ch:conclusie}, tenslotte, wordt de conclusie gegeven en een antwoord geformuleerd op de onderzoeksvragen. Daarbij wordt ook een aanzet gegeven voor toekomstig onderzoek binnen dit domein.
\chapter{\IfLanguageName{dutch}{Stand van zaken}{State of the art}}%
\label{ch:stand-van-zaken}

% Tip: Begin elk hoofdstuk met een paragraaf inleiding die beschrijft hoe
% dit hoofdstuk past binnen het geheel van de bachelorproef. Geef in het
% bijzonder aan wat de link is met het vorige en volgende hoofdstuk.

% Pas na deze inleidende paragraaf komt de eerste sectiehoofding.

In dit hoofdstuk zal de literatuurstudie besproken worden. Door deze literatuurstudie is het mogelijk om een beter inzicht te krijgen in de technologie en mogelijkheden voor de documentatie van Python projecten. 
Alsook hoe het toegepast kan worden met behulp van Large Language Modellen. Er
zal nadruk worden gelegd op bestaande literatuur en onderzoeken die verbonden
zijn met documentatie van Python projecten. In dit onderdeel zullen verschillende hoofdstukken worden aangekaart. 
Als eerste zal er duidelijk gemaakt worden wat er juist verstaan wordt met documentatie. 
Dan wordt er gekeken naar wat Large Language Modellen zijn, hoe deze werken en wat enkele bestaande modellen zijn.
Vervolgens wordt er gekeken naar bestaande documentatie tools.
Als laatste wordt er gekeken naar hoe Large Language Modellen gebruikt kunnen worden voor het genereren van documentatie.

\section{Wat is documentatie?}
\label{sec:wat-is-documentatie}

Voor dat er dieper op het onderwerp wordt ingegaan is het belangrijk dat er een duidelijk beeld is van wat documentatie is. 
Waarom is documentatie belangrijk voor een project en wat wordt er begrepen onder documentatie? 

Documentatie is het proces van het vastleggen van de werking van een project.
Dit kan op verschillende manieren gebeuren. 
Er kan gekozen worden om de documentatie te schrijven in de vorm van een handleiding, een wiki, een website of in de vorm van commentaar in de code.
Het doel van documentatie is om de werking van het project te beschrijven zodat andere programmeurs het project kunnen begrijpen en gebruiken.
Zodat er geen tijd verloren gaat aan het lezen van de code en het begrijpen ervan.

Documentatie kan gemaakt worden voor verschillende doelgroepen. Het kan voor interne of externe doeleinden zijn.
Interne documentatie is voor documentatie binnen hetzelfde bedrijf.
Dit gaat dan om het capteren van de process kennis die vergaard is binnen een project, dit is informatie zoals een roadmap of product requirements. 
Of het gaat over het vastleggen van gedetaileerde uitleg over hoe iets werkt en hoe het onderhouden kan worden.

Externe documentatie is voor documentatie die gedeeld wordt met andere bedrijven of klanten. 
Dit gaat dan over de basis werking van de code van een project zodat andere programmeurs het kunnen gebruiken.
Gebruiksaanwijzingen of handleidingen zijn ook een vorm van externe documentatie. \autocite{swimm.io2024}

Voor deze bachelorproef wordt er gekeken naar het documenteren van een Python project.
In de vorm van commentaar in de code en het genereren van een samenvattend document van het gehele project.
Ook kan er in de code bij functies aan typehinting gedaan worden, dit indiceert wat de datatypes van de input en output van een functie zijn \autocite{Bailey2024}.
Waaruit de werking van het project duidelijk wordt en de relatie tussen de verschillende bestanden en functies.

\section{Wat zijn Large Language Modellen (LLM)?}
\label{sec:wat-zijn-llms}

Omdat er in deze bachelorproef gebruik gemaakt wordt van Large Language Modellen is het belangrijk dat er een duidelijk beeld is van wat deze modellen zijn.
Wat kunnen deze modellen, wat zijn de mogelijke beperkingen en wat is de huidige stand van zaken. 
In dit hoofdstuk wordt er een antwoord gegeven op de vragen: Bestaan er LLMs speciaal getraind op Python code? Of kunnen LLMs gebruikt worden om documentatie te genereren?
Dit zorgt voor een grondige basiskennis van LLMs. 

Het veld waarin AI zich bevindt wordt vaak voorgesteld volgens figuur \ref{fig:LLM-position}, met verschillende lagen \autocite{Stoeffelbauer2023}.
Deze lagen zijn: Artificiële Intelligentie, Machine Learning, Deep Learning en Large Language Modellen.
Voor dat er dieper op de LLMs wordt ingegaan is het belangrijk dat er een duidelijk beeld is van wat deze lagen juist inhouden.

\begin{figure}[h]
  \centering
  \includegraphics[width=0.5\textwidth]{LLMsphere.png}
  \caption{Artificiële intelligentie in lagen \autocite{Stoeffelbauer2023}}
  \label{fig:LLM-position}
\end{figure}

AI is een brede term, hiermee wordt vaak verwezen naar slimme machines. 
Machine Learning (ML) is een subveld van AI, waarin patronen worden herkend tussen een input en een output.
ML kan gebruikt worden voor verschillende taken zoals: classificatie, regressie, clustering \dots
Deep Learning (DL) is een subveld van ML, waarin complexe algoritmen en deep neural networks gebruikt worden om moeilijkere taken uit te voeren.
Deep Learning is een krachtige tool die gebruikt wordt voor verschillende taken zoals: beeldherkenning, spraakherkenning, \dots \autocite{Stoeffelbauer2023}.

Large Language Modellen zijn geavanceerde AI-systemen die dienen om menselijke taal te verstaan, te genereren en te verwerken.
LLMs worden getraind op een grote hoeveelheid tekst wat vaak uit allerlei data zoals artikels of websites gehaald wordt. 
Volgens \textcite{Beelen2023} zorgen deep neural networks ervoor dat LLM's natuurlijke taal verwerken op een gelijkaardige manier die vergelijkbaar is met de menselijke taalvaardigheid.
Deze hebben een grote vooruitgang gekend in 2017 door de paper van \textcite{VaswaniEtAl2017}. 
Hieruit kwam een nieuw mechanisme tot stand namelijk transformers wat bestaat uit Attention blokken. 
Enkele voordelen die komen kijken bij het gebruiken van transformers zijn: het kan lange sequenties verwerken, het kan parallel verwerken en het kan de relaties tussen de verschillende delen van de sequentie leren.
Hierdoor hebben transformer modellen een snellere trainings periode dan vorige neurale netwerken \autocite{aiml2023}.

\subsection{Transformers en de architectuur van LLMs}
\label{sec:architectuur-van-llms}
Een neuraal netwerk bestaat uit verschillende lagen. Enkele belangrijke blokken die gebruikt worden binnen de transformer laag zijn:
\begin{itemize}
  \item Self-Attention
  \item Cross-Attention
  \item Masked Self-Attention
\end{itemize}

Deze attention blokken worden gebruikt in de encoder en decoder van een transformer en stromen voort uit het onderzoek van \textcite{VaswaniEtAl2017}.

Transformers zijn een speciaal type van neurale netwerken die gebruik maken van verschillende attention blokken.
Attention is een mechanisme dat gebruikt wordt om de relaties tussen verschillende delen van de invoersequenties te leren.
Een transformer bestaat uit een encoder en een decoder. 
Niet elke transformer bestaat uit beide een encoder en een decoder, sommige bestaan enkel uit een encoder of een decoder \autocite{Hoque2023}.
De encoder wordt gebruikt om de invoersequenties te verwerken en de decoder wordt gebruikt om de uitvoersequenties te genereren.
Zo is BERT van \textcite{DevlinEtAl2019} een transformer die enkel een encoder heeft en GPT van \textcite{RandfordEtAL2018} heeft enkel een decoder.
De transformer architectuur uit de paper van \textcite{VaswaniEtAl2017} kan gezien worden in figuur \ref{fig:transformer-model}. 

Self-Attention duidt dynamisch gewichten toe aan verschillende elementen binnen de meegegeven sequentie, bijvoorbeeld woorden in een zin.
Dit laat het model toe om zich te concentreren op de meest relevante delen van de invoer, terwijl de invloed van minder cruciale delen wordt verminderd.
De invoersequentie wordt eerst in drie verschillende vectoren omgezet: query, key en value.
De Query vector stelt een specifiek token uit de invoersequentie voor, de Key vector vertegenwoordigt alle tokens en de vector voor Value bevat de feitelijke inhoud die aan elk token is gekoppeld.
De similariteit tussen de Query en de Key vector wordt berekend aan de hand van het inwendig product van de twee vectoren.
Deze similariteit wordt gebruikt om de gewichten te berekenen die aan de Value vector worden toegekend \autocite{VaswaniEtAl2017}.

Masked Self-Attention is een variant van Self-Attention die gebruikt wordt in de decoder van een transformer.
In de decoder wordt er een mask gebruikt om enkel de vorige tokens te zien in de sequentie \autocite{VaswaniEtAl2017}.
Dit vermijdt dat er informatie van de toekomstige tokens gebruikt wordt. 
Zo kan de transformer niet "vals spelen" tijdens het train proces.

Cross-Attention is een variant van Self-Attention die gebruikt wordt in de decoder van een transformer.
Deze laag gebruikt de informatie van de encoder en de vorige Attention laag van de decoder om de uitvoersequenties te genereren.
De query vector is de uitvoer van de vorige Attention/Cross-Attention laag van de decoder en de key en value vector zijn de uitvoer van de encoder \autocite{VaswaniEtAl2017}.
Doordat de Cross-Attention laag informatie van zowel de encoder als decoder krijgt kan het model de relaties tussen de verschillende delen van de invoersequenties leren.
Deze relaties worden dan gebruikt om de uitvoersequenties te genereren.

\begin{figure}[h]
  \centering
  \includegraphics[width=0.5\textwidth]{transformer.png}
  \caption{Transformer - model architectuur \autocite{VaswaniEtAl2017}}
  \label{fig:transformer-model}
\end{figure}

\subsection{Trainen van LLMs}
\label{sec:trainen-van-llms}
Het trainen van LLMs is een complex proces dat veel tijd en rekenkracht vereist. Dit gebeurt in verschillende stappen.
De eerste fase begint bij het verzamelen van een grote hoeveelheid tekst die gebruikt wordt om het model te trainen.
Deze tekst wordt gehaald uit verschillende artikelen, websites, boeken, \dots Zo kan het volgende woord in een sequentie van tekst voorspeld worden.

Het model krijgt deze grote hoeveelheid tekst in de pre-training fase.
In deze fase leert de LLM grammatica, semantiek, taal patronen en factuele informatie \autocite{Cacic2023}.
Voordat de data meegegeven wordt aan het model moet de data gecleaned en geformatteerd worden.
Dit gebeurt in het tokenization process. Hier wordt de tekst omgezet in tokens die het model kan verwerken.
Woorden kunnen kleiner gemaakt worden zodat de volledige tekst in het model past, dit moet wanneer het model een beperkte input capaciteit heeft \autocite{ElHousieny2023}.
Wat dan op zijn beurt omgezet wordt in embeddings. Deze embeddings worden dan meegegeven aan het model om te trainen.
\begin{figure}[h]
  \centering
  \includegraphics[width=0.5\textwidth]{tokenization.png}
  \caption{Simplified tokenization van tekst \autocite{TeeTracker2023}}
  \label{fig:tokenization}
\end{figure}
Uit de data kunnen dan patronen gehaald worden met behulp van Transformers \ref{sec:architectuur-van-llms}, maar het is nog niet instaat om vragen of instructies te begrijpen.

De volgende fase is het model te trainen op een dataset met instructies en het antwoord erop, dit is het gesuperviseerde Fine-Tuning van de LLM \autocite{Das2024}. 
Het model probeert zo de patronen te leren die nodig zijn om vragen te beantwoorden of instructies te volgen.
Hierdoor leert het model instructies te volgen en vragen te beantwoorden.

Er kan gebruik gemaakt worden om het model specifiek aan de wensen van de mens te laten voldoen. Dit kan door het gebruiken van Reinfocement Learning met menselijke feedback \autocite{LambertEtAL2022}. 
Hierbij geeft de mens feedback aan het model en leert het model bij door deze feedback.

Het model kan achteraf nog extra getraind worden op specifieke data. 
Het Fine-Tunen van het model kan gebeuren op een specifieke dataset, zoals Python code of medische data.
Dit zorgt ervoor dat het model extra kennis heeft over het gekozen onderwerp.

\subsection{Bestaande LLMs}
\label{sec:bestaande-llms}

Momenteel zijn er verschillende LLMs die gebruikt worden voor verschillende taken.
Deze LLMs zijn getraind op verschillende data en hebben verschillende architecturen.
Het is belangrijk dat er een duidelijk beeld is van de verschillende LLMs en hun mogelijkheden. 
Zodat er een goede keuze gemaakt kan worden voor het genereren van documentatie.

Eén van de grote spelers in de wereld van LLMs is OpenAI. OpenAI heeft verschillende LLMs ontwikkeld gaande van GPT \autocite{RandfordEtAL2018} tot GPT-4 \autocite{OpenAI2023}.
GPT-4 is een LLM die OpenAI heeft ontwikkeld, het is getraind op een grote hoeveelheid data en heeft een grote capaciteit.
Een nadeel is dat GPT-4 een betalende service is \autocite{OpenAI2023}.

Een andere grote speler is Google, Google heeft verschillende LLMs ontwikkeld waaronder BERT van \textcite{DevlinEtAl2019} en Gemini \autocite{Google2024}
BERT staat voor Bidirectional Encoder Representations from Transformers, een DL model waar elk output element verbonden is met elk input element \autocite{HashemiPour2024}.
BERT was een eerste stap in de wereld van LLMs voor Google. Sinds kort heeft \textcite{Google2024} een nieuwe LLM ontwikkeld genaamd Gemini.
Deze LLM is een sterke concurent voor GPT-4 van \textcite{OpenAI2023}. Het bestaat uit verschillende versies: Gemini Pro, Gemini Ultra en Gemini Nano. 
Elke versie is gemaakt voor een specifiek doeleind, zo is Gemini Nano het meest efficiente model voor mobiele toestellen. Gemini Pro is dan weer het beste model voor het schalen van allerlei taken.
En Gemini Ultra is het meest capabele en grootste model van Google, dit kan gebruikt worden voor complexe taken.
Een van de voordelen van Gemini is dat er een groot aantal input tokens meegegeven kunnen worden, namelijk 1 miljoen tokens.
Dit is aanzienlijk meer dan de 128 duizend tokens van GPT-4.

Een derde speler in de wereld van LLMs is Meta. Meta heeft verschillende LLMs ontwikkeld onder de naam LLama 2 \autocite{Meta2024}.
De LLama 2 familie bestaat uit verschillende LLMs die getraind zijn op verschillende data. Sommige zijn extra getraind voor specifiekere doeleinden.
Zo is er bijvoorbeeld een LLM getraind op Python code, genaamd Code LLama 2 van \textcite{Roziere2024}.
Een voordeel van de LLama 2 familie is dat deze LLMs open source zijn en dus voor iedereen toegankelijk zijn.

Antropic heeft ook een LLM ontwikkeld genaamd Claude \autocite{Anthropic2023}. 
Claude's capaciteiten zijn code generatie, het verstaan van meerdere talen, beelden analyseren en kan geavanceerde redeneringen geven.
Er bestaan 3 versies van Claude: Haiku, Sonnet en Opus.
Haiku een lichte versie van Claude, Sonnet is de combinatie van performantie en snelheid en Opus is het intelligentste model dat complexe taken kan uitvoeren en begrijpen.
Claude is een betalende service, de prijzen zijn afhankelijk van de gekozen versie van Claude.

De verschillen tussen deze LLMs zijn groot, zo is er een verschil in capaciteit, trainingsdata en toegankelijkheid.
Het is belangrijk dat er een goede keuze gemaakt wordt voor het genereren van documentatie.
Deze keuze zal afhangen van de mogelijkheden van de LLMs en de doeleinden van de documentatie.
Het is mogelijk dat er meerdere LLMs getest moeten worden om de beste keuze te maken.

\section{Bestaande documentatie tools}
\label{sec:huidige-tools}
Voor er gekeken wordt naar hoe LLMs mogelijk gebruikt kunnen worden voor het genereren van documentatie is het belangrijk dat er een duidelijk beeld is van de huidige tools die gebruikt worden voor het genereren van documentatie.
De documentatie kan in verschillende vormen gegeneerd worden dit kan gaan van een website tot een samenvattend document.
Ook kunnen er in de code zelf commentaren geplaatst worden die de werking van de code uitleggen.
Hiervoor bestaan er reeds verschillende tools en dit voor verschillende programmeertalentalen.

In de paper van \textcite{SridharaEtAL2010} werd er een tool ontwikkel die natuurlijke taal genereert op basis van JAVA code. 
Het selecteert eerst de relevante code en genereert dan een samenvatting van de code, volgens enkele programmeurs die de tool getest hebben was de samenvatting correct en volledig.

Doxygen \autocite{Doxygen2023} is een tool die het toelaat om automatisch code documentatie te genereren. Het is een gratis tool die bruikbaar is voor verschillende programmeertalen zoals: C++, C, Python, PHP en Java.
Het genereert documentatie in de vorm van HTML, LaTeX, RT. Ook is het in staat om een diagram te genereren met de relaties tussen de verschillende delen van de code. 
Bijvoorbeeld de relaties tussen de verschillende klassen en functies.
Een voorbeeld van een diagram kan gezien worden in figuur \ref{fig:Doxygen-diagram}.
Zo wordt er een duidelijk beeld verkregen van de structuur van het project.

\begin{figure}[h]
  \centering
  \includegraphics[width=1\textwidth]{doxygen_diagram.png}
  \caption{Voorbeeld diagram van Doxygen \autocite{Doxygen2023}}
  \label{fig:Doxygen-diagram}
\end{figure}

Docstrings is een vorm van commentaar in de code die gebruikt wordt om de werking van de code uit te leggen.
Deze commentaren worden gebruikt om de werking van een module, functie, klasse of methode uit te leggen.
Dit kan automatisch gegenereerd worden met tools zoals CodeCat \autocite{CodeCat2024} voor JavaScript.
\textcite{CodeCat2024} is een online tool die de code analyseert en de docstrings genereert, het is niet open sourced dus er kan niet gekeken worden naar de werking van de tool.

De tool van \textcite{Trofficus2023} doet dit voor Python, het maakt gebruik van GPT-4 \autocite{OpenAI2023} om de docstrings te genereren.
Deze tool leunt sterk aan bij de doelstelling van deze bachelorproef, namelijk het genereren van documentatie met behulp van LLMs.
Het nader bekijken van deze tool kan een meerwaarde zijn voor deze bachelorproef.

\begin{figure}
  \centering
  \begin{subfigure}[b]{0.5\textwidth}
      \centering
      \includegraphics[width=1\textwidth]{before_troficus.png}
      \caption{Voorbeeld code zonder docstrings van \textcite{Trofficus2023}}
      \label{fig:before-Trofficus}
  \end{subfigure}
  \hfill
  \begin{subfigure}[b]{0.5\textwidth}
      \centering
      \includegraphics[width=1\textwidth]{after_troficus.png}
      \caption{Voorbeeld code met docstrings van \textcite{Trofficus2023}}
      \label{fig:after-Trofficus}
  \end{subfigure}
     \caption{Voorbeeld uitkomst van de tool van \textcite{Trofficus2023}}
     \label{fig:Before-After-Trofficus}
\end{figure}

Zo gebruikt het de Abstract Syntax Tree (AST) van de code om de structuur van de code te begrijpen en vast te nemen.
Uit de AST kunnen de juiste stukken code gehaald worden om de docstrings te genereren.
Dit kan goed van pas komen voor het genereren van documentatie van Python projecten.
Een voorbeeld van deze tool kan gezien worden in figuur \ref{fig:Before-After-Trofficus}. 

\textcite{Sphinx2023} is een van de meest gebruikte tools voor het genereren van documentatie voor Python projecten.
Het genereert documentatie aan de hand van docstrings en de hierarchie van het project om een duidelijk overzicht te geven.
Deze tool is vrij flexibel want het kan uitgebreid worden met verschillende extensies, zodat het alle mogelijke wensen kan vervullen.
Bijvoorbeeld de extensie autodoc kan semi-automatisch de docstrings van een module extraheren en in de documentatie plaatsen. 
Handig wanneer de automatische documentatie generatie van een geheel project gewenst is, zo kan het project samen gevat worden aan de hand van de docstrings van de verschillende python files. 
Eer een Python project gedocumenteerd kan worden met Sphinx dienen alle bestanden aangevuld te worden met docstrings, dit gebeurt echter niet automatisch.

Pdoc \autocite{GallantHils2023} genereert documentatie in de vorm van een website die een API van de documentatie bevat. 
Hier kan er makkelijk op de website gezocht worden naar een functie of klasse met de bijhorende documentatie.

\section{LLM voor documentatie}
\label{sec:llm-voor-documentatie}

Nu er geweten is hoe een LLM werkt en wat het doet. Wat enkele bekende LLMs zijn en wat hun mogelijkheden zijn. 
En wat enkele bestaande documentatie tools zijn.
Is het belangrijk om te kijken naar hoe LLMs gebruikt kunnen worden voor het genereren van documentatie.
Dit kan staps gewijs gebeuren, eerst kunnen de verschillende delen van het project meegegeven worden aan de LLM. 
Hier kan er telkens aan de Large Language Model gevraagd worden om een samenvatting te maken van wat dit deel van het project doet en wat de uitkomst is.
Door dit te herhalen voor alle files van het project kan er achteraf één samenvattend document gemaakt worden van het gehele project.

Ook kan er gevraagd worden aan de LLM om de relatie tussen de verschillende delen van het project te beschrijven in de samenvattingen.
Zo kan er een duidelijk beeld verkregen worden van de structuur van het project en hoe de verschillende delen van het project samenwerken.
Alle functies van het python project kunnen hier makkelijk teruggevonden worden.

Er kan ook gebruik gemaakt worden van de LLM om de docstrings van de verschillende functies en klassen te genereren. 
Om zo een betere samenvatting te verkrijgen van de werking van de verschillende delen van het project, door de docstrings te combineren met de samenvattingen van de LLM.
Door telkens de docstrings en de naam van de verschillende functies en klassen mee te geven aan de LLM kan er een betere samenvatting gemaakt worden van het gehele project.

Wanneer dat een huidige LLM niet instaat is om de gewenste documentatie te genereren kan een LLM gefinetuned op specifieke data, Python code en de bijhorende documentatie.
Hier is er een grote hoeveelheid data van Python projecten met de bijhorende documentatie nodig. Ook is het duur en tijdrovend om een LLM te finetunen.
%%=============================================================================
%% Methodologie
%%=============================================================================

\chapter{\IfLanguageName{dutch}{Methodologie}{Methodology}}%
\label{ch:methodologie}

%% TODO: In dit hoofstuk geef je een korte toelichting over hoe je te werk bent
%% gegaan. Verdeel je onderzoek in grote fasen, en licht in elke fase toe wat
%% de doelstelling was, welke deliverables daar uit gekomen zijn, en welke
%% onderzoeksmethoden je daarbij toegepast hebt. Verantwoord waarom je
%% op deze manier te werk gegaan bent.
%% 
%% Voorbeelden van zulke fasen zijn: literatuurstudie, opstellen van een
%% requirements-analyse, opstellen long-list (bij vergelijkende studie),
%% selectie van geschikte tools (bij vergelijkende studie, "short-list"),
%% opzetten testopstelling/PoC, uitvoeren testen en verzamelen
%% van resultaten, analyse van resultaten, ...
%%
%% !!!!! LET OP !!!!!
%%
%% Het is uitdrukkelijk NIET de bedoeling dat je het grootste deel van de corpus
%% van je bachelorproef in dit hoofstuk verwerkt! Dit hoofdstuk is eerder een
%% kort overzicht van je plan van aanpak.
%%
%% Maak voor elke fase (behalve het literatuuronderzoek) een NIEUW HOOFDSTUK aan
%% en geef het een gepaste titel.


Het onderzoek is in vier fases opgedeeld. De eerste fase omvat de literatuurstudie.
In deze literatuurstudie wordt er onderzocht wat de huidige stand van zaken is omtent de technologie en mogelijkheden voor de documentatie van Python projecten met behulp van Large Language Modellen.
Zo wordt er gekeken naar wat LLMs zijn, hoe ze werken en wat bestaande tools zijn voor het genereren van documentatie.

Nadat er een duidelijk beeld gevormd is in de literatuurstudie, kan er begonnen worden aan de tweede fase.
Hier wordt er een tool ontwikkeld die een Python bestand kan analyseren en op basis daarvan documentatie kan genereren, aan de hand van het toevoegen van docstrings aan de code.
Dit doet het eerst per functie dan per klasse en uiteindelijk voor het gehele bestand. Dit kan later gebruikt worden om een samenvatting van het project te genereren in de volgende fase.
De uitkomst van deze fase is een tool die de documentatie van een Python bestand kan genereren. 
Door aan prompt engineering te doen, met het prompt dat meegegeven wordt aan de LLM, kunnen de bekomen docstrings accurater worden. %% Accuraatheid?? testen??

In de derde fase wordt er gekeken naar hoe de documentatie van Python functies gebruikt kan worden voor het maken van een gehele samenvatting van het project.
Dit wordt gedaan op basis van huidige methoden om docstrings aan te maken en te gebruiken. Erna kunnen de verschillende docstrings en naam van de functie of klasse gebruikt worden om een samenvatting te genereren.
Deze informatie kan dan gegeven worden aan de Large Language Modellen om een samenvatting te genereren.




% Voeg hier je eigen hoofdstukken toe die de ``corpus'' van je bachelorproef
% vormen. De structuur en titels hangen af van je eigen onderzoek. Je kan bv.
% elke fase in je onderzoek in een apart hoofdstuk bespreken.

% ==============================================
% Resultaten
% ==============================================
\chapter{\IfLanguageName{dutch}{Resultaten}{Results}}%
\label{ch:resultaten}

% =================================================================================================
% File Documentation
% =================================================================================================
\section{Bestand documentatie}
\label{sec:bestanddocumentatie}

\subsection{Inleiding}
\label{sec:bestanddocumentatie-inleiding}
In dit hoofdstuk wordt er gekeken naar de documentatie van een Python bestand.
Eerst wordt de code van het bestand geanalyseerd en worden de verschillende functies en klassen geïdentificeerd.
Op basis van deze functies en klassen worden er docstrings gegenereerd, die opnieuw toegevoegd worden aan de code van het bestand.
Daarna worden de docstrings binnen het bestand gebruikt om een samenvatting van het bestand te genereren.

Deze samenvatting kan dan als basis gebruikt worden voor het genereren van documentatie voor een Python project, wat bestaat uit een gehele samenvatting en de relaties tussen de verschillende bestanden van het project.
Dit wordt verder onderzocht in het volgende hoofdstuk.
Voor dit kan gebeuren, moet de bestand documentatie op punt staan en geoptimaliseerd worden.

Omdat dit onderzoek een bepaalde scope heeft, is er gekozen om enkel te kijken naar het documenteren van correcte Python bestanden.
Dit wil zeggen dat er verwacht wordt dat de code correct is en dat er geen syntax fouten in de code zitten.
Er wordt niet gekeken naar het documenteren van bestanden met syntax fouten of bestanden die niet correct zijn.

\subsection{GPT4Docstrings tool}
\label{sec:bestanddocumentatie-tool}
Uit de literatuurstudie is gebleken dat de tool GPT4Docstrings van \textcite{Trofficus2023} een tool is die het toelaat om docstrings te genereren voor ongedocumenteerde Python projecten.
Deze tool maakt gebruik van GPT4 \autocite{OpenAI2023} om de docstrings te genereren.
De resultaten van deze tool zijn beoordeeld en geëvalueerd volgens de requirementsanalyse, een voorbeeld van deze uitkomst is te zien in \ref{lst:gpt4docstrings-uitkomst}.

\begin{listing}
    \caption{Voorbeeld uitkomst van GPT4Docstrings. \autocite{Trofficus2023}}
    \label{lst:gpt4docstrings-uitkomst}
    \begin{minted}{python}
import asyncio
async def async_example():
    """
    An asynchronous example function.

    This function asynchronously sleeps for 2 seconds.

    Returns
    -------
    None
        This function does not return any value.
    """
    await asyncio.sleep(2)
    \end{minted}
\end{listing}

De tool is enkel instaat om docstrings te genereren voor functies en klassen.
Hierdoor is er gekozen om niet de tool zelf te gebruiken maar om de technieken die te tool gebruikt te implementeren in een eigen tool.
Dit laat toe om de tool uit te breiden en te verbeteren volgens de requirementsanalyse, ook geeft het meer controle over de gegenereerde docstrings.

\subsection{Abstract Syntax Tree}
\label{sec:bestanddocumentatie-ast}
Voor er docstrings gegenereerd kunnen worden, moet er eerst gekeken worden naar hoe de code van een Python bestand geanalyseerd kan worden.
Uit de literatuurstudie is gebleken dat de verschillende functies en klassen in een bestand geïdentificeerd en geëxtraheerd kunnen worden aan de hand van een Abstract Syntax Tree (AST).
Het analyseren van de code van de tool GPT4Docstrings gemaakt door \textcite{Trofficus2023} heeft een beter beeld gegeven van hoe een AST eruit ziet en hoe deze gegenereerd kan worden.

Een AST is een boomstructuur die de syntactische structuur van een programma weergeeft.
Per knoop in de boom wordt er een deel van de code voorgesteld. 
Deze knoop kan dan weer kinderen hebben die deel uitmaken van de code.
Elke knoop in de boom heeft een type en een waarde.

Het inlezen van een Python bestand en deze omzetten naar een AST, maakt het mogelijk om de code van het bestand te manipuleren.
Zo kunnen de verschillende import statements, functies en klassen geïdentificeerd worden.

\begin{listing}
    \caption[Ophalen functies uit AST]{Voorbeeld van het ophalen van functies uit een AST.}
    \label{lst:ast-voorbeeld}
    \begin{minted}{python}
        def get_functions(self):
            functions = {}
            for node in ast.walk(self.tree):
                if isinstance(node, ast.FunctionDef) or isinstance(node, ast.AsyncFunctionDef):
                    function_code = ast.unparse(node)
                    functions[node.name] = function_code
            return functions
    \end{minted}
\end{listing}

In \ref{lst:ast-voorbeeld} wordt er met behulp van de ast.walk functie door de AST gelopen.
Elke node in de AST wordt gecontroleerd of het een functie of een asynchrone functie is.
Als dit het geval is, wordt de code van de functie opgehaald en toegevoegd aan een dictionary.

\subsection{Docstrings}
\label{sec:bestanddocumentatie-docstrings}
Binnen deze bachelorproef wordt de docstring stijl van Google gehanteerd \autocite{GPT2024}.
Deze docstrings bestaan uit een korte beschrijving van de functie of klasse, de argumenten die de functie verwacht en de return waarde van de functie.
Een voorbeeld van een docstring voor een functie die controleert of een getal een priemgetal is \ref{lst:docstring-voorbeeld}.

\begin{listing}
    \caption[Docstring van een functie]{Voorbeeld van een docstring voor een functie die controleert of een getal een priemgetal is.}
    \label{lst:docstring-voorbeeld}
    \begin{minted}{python}
    def is_prime(n: int) -> bool:
        """
        Check if a number is prime.

        Args:
            n (int): The number to check.

        Returns:
            bool: True if the number is prime, False otherwise.
        """
    \end{minted}
\end{listing}

Deze docstrings dienen gegenereerd te worden voor elke functie en klasse in een Python bestand op basis van de huidige code van de functie of klasse.

\subsection{Keuze van model}
\label{sec:bestanddocumentatie-model}

Uit de literatuurstudie is gebleken dat GPT3.5-Turbo van \textcite{OpenAi2024} het beste model is. 
Dit model heeft de beste prijs-kwaliteit verhouding volgens de tabellen \ref{table:vgl-llms-eval} en \ref{table:vgl-llms} van de literatuurstudie.
Dit is een krachtig model dat getraind is op een grote hoeveelheid data en in staat is om natuurlijke taal te genereren.
Aangezien dit model getraind is op een grote hoeveelheid data is het in staat om met de juiste prompts de gewenste uitkomst te genereren.

\subsection{Prompting}
\label{sec:bestanddocumentatie-prompting}

Door aan prompt-engineering te doen kan het model beter aangestuurd worden en kan de gewenste uitkomst volgens de requirementsanalyse bekomen worden.
Er werden verschillende prompts getest om het beste resultaat te bekomen.
Er zijn prompts gemaakt voor het genereren van docstrings voor functies en klassen. 

\subsubsection{Prompt engineering voor functies}

\begin{listing}
    \caption{Prompt voor het genereren van een docstring voor een functie v1.}
    \label{lst:prompt1}
    \begin{minted}{python}
        '''For this Python function:
        ```python	
        def is_prime(n):
        if n in [2, 3]:
            return True
        if (n == 1) or (n % 2 == 0):
            return False
        r = 3
        while r * r <= n:
            if n % r == 0:
                return False
            r += 2
        return True
        ```
        Leave out any imports, just return the function with the docstring and type hints.
        The function, with docstring using the google docstring style and with type hints is:
        ```python	
        def is_prime(n: int) -> bool:
        """
        Check if a number is prime.
        Args:
            n (int): The number to check.
        Returns:
            bool: True if the number is prime, False otherwise.
        """
        if n in [2, 3]:
            return True
        if (n == 1) or (n % 2 == 0):
            return False
        r = 3
        while r * r <= n:
            if n % r == 0:
                return False
            r += 2
        return True
        ```
        
        For this Python function:
        ```python	
        {code}
        '''
    \end{minted}
\end{listing}

Het eerste prompt voor het genereren van een docstring voor een functie is te zien in \ref{lst:prompt1}.
Er wordt gevraagd om een docstring te genereren voor een functie. Dit prompt bevat een voorbeeld van een functie met de verwachte uitkomst. 

De volgende versie van dit prompt bevat duidelijkere instructies \ref{bijlage:prompt2}.
Het is belangrijk dat de prompt duidelijk is en dat het model weet wat er verwacht wordt. 
In de instructies staat exact wat er verwacht wordt van het model. Dat de gegenereerde functie een docstring moet bevatten en type hints.
De code van de functie mocht niet aangepast worden en er mochten geen imports toegevoegd worden.
Ook mocht het model niets veronderstellen over de functie of de data types die gebruikt worden in de functie.

Door het vergelijken van de uitkomst \ref{lst:uitkomst-prompt2} met een vooropgestelde uitkomst \ref{bijlage:zelfgedocumenteerd-bestand}, een zelfgedocumenteerd bestand, kon er gekeken worden of de gegenereerde docstrings correct waren.
\begin{listing}
    \caption{Uitkomst prompt voor het genereren van een docstring voor een functie v2.}
    \label{lst:uitkomst-prompt2}
    \begin{minted}{python}
    def crop_faces(plot_images: bool=False, max_images_to_plot: int=5) -> List[ndarray]:
    """
    Crop faces from images and save them in a directory.

    Args:
        plot_images (bool, optional): Whether to plot the images. Defaults to False.
        max_images_to_plot (int, optional): Maximum number of images to plot. Defaults to 5.

    Returns:
        List[ndarray]: List of cropped images.
    """
    \end{minted}
\end{listing}
Hier is te zien dat de uitkomst van het model correct is buiten dat de type hints niet correct zijn. 
Dit probleem is opgelost door met de volgende versie \ref{bijlage:prompt3} van de prompt de import statements mee te geven.
Zo kan het model geen foute veronderstellingen maaken ook al werd er in de instructies duidelijk meegegeven dat dit niet de bedoeling was.

Deze veronderstellingen kwamen er omdat het model de code van de functie sporadisch aanpaste.
In het aangepaste prompt werd er duidelijk gemaakt dat de uitkomst van de prompt slechts de functienaam met typehint en de docstring moest bevatten.
De code van de functie moest niet meer in de uitkomst staan.

\subsubsection{Prompt engineering voor klassen}
de prompt engineering process voor klassen liep gelijkaardig met dat van functies.
Er werd een prompt gemaakt met duidelijke instructies en een voorbeeld van een klasse met de verwachte uitkomst.
De instructies waren gelijkaardig aan die van de functies, maar dan voor klassen.

De verschillende prompt versies 1-4 voor klassen zijn identiek aan die van functies, maar dan met de code van een klasse in plaats van een functie.
Omdat een klasse bestaat uit verschillende functies en attributen is het belangrijk dat de docstrings van de functies en attributen correct gegenereerd worden.
Deze docstrings worden dan meegegeven in de prompt voor het genereren van de docstring van de klasse.
Hiervoor is de overige code van de klasse niet nodig, wat dan ook niet meegegeven wordt in de prompt.
Ook worden opnieuw de verschillende imports meegeven, aangezien deze hallucinaties vermijden zoals gezien in \ref{sec:bestanddocumentatie-prompting}.
Door de uitkomst \ref{lst:uitkomst-prompt4} van prompt versie 4 \ref{bijlage:prompt4} te vergelijken met een vooropgestelde uitkomst \ref{bijlage:zelfgedocumenteerd-bestand-2} kon er gekeken worden of de gegenereerde docstrings correct waren.
Hier is te zien dat de uitkomst overeenkomt met de vooropgestelde uitkomst. 

\begin{listing}
    \caption{Uitkomst prompt voor het genereren van een docstring voor een klasse v4.}
    \label{lst:uitkomst-prompt4}
    \begin{minted}{python}
    class CsvReader:
    """
    A class representing a CSV file reader with a method to read the file and return its content as a list of rows.

    Methods:
        readCsv: Read a CSV file and return its content as a list of rows.
    """
    \end{minted}
\end{listing}

\subsubsection{Prompt engineering voor samenvatting}
Voor het genereren van een samenvatting van een Python bestand, werd er een prompt gemaakt met de gegenereerde docstrings van de functies en klassen.
De verschillende docstrings werden meegegeven aan de parameter \mintinline{python3}|code_content| en de naam van het bestand aan de parameter 
\mintinline{python3}|filename|.
Het volledige prompt met alle beschrijvingen kan gevonden worden in \ref{bijlage:bestand-samenvatting}.
Door de gekende technieken van prompt engineering gezien in \ref{sec:prompt-engineering} te grbruiken, kan het model aangestuurd worden.
Er wordt meegegeven wat er verwacht wordt van het model, wat er in de uitkomst moet staan en op basis van welke data de uitkomst gegenereerd moet worden.

\subsection{Toevoegen van gegenereerde docstrings}
\label{sec:bestanddocumentatie-vervangen}
De gegenereerde docstrings worden vervolgens toegevoegd aan de code van de functies en klassen.
Dit gebeurt door de code van de functie of klasse te vervangen door de gegenereerde docstring.
Deze kunnen vastgelegd worden in de AST en dan de AST gebruiken als de nieuwe code van het bestand.

Omdat de uitkomst van de prompts altijd in de vorm van een string met een omsloten code blok wordt gegeven, zoals te zien in \ref{bijlage:prompt1}.
Dient dit weg gehaald te worden voor het toevoegen aan de code van het bestand. 
Dit wordt gedaan door de uitkomst van het model te parsen en de code blokken te verwijderen, wat behouden moet worden is de gegenereerde docstring samen met de functie of klasse declaratie.

De eerste versie van de code voor het toevoegen van de docstrings aan de code van de functies en klassen is te zien in \ref{lst:vervangen-v1}.
Door te testen en evalueren met verschillende Python bestanden met moeilijkheidsgraden zoals te zien in \ref{table:bestanden} werd er gekeken of de code correct werkte.

\begin{table}[h!]
    \centering
    \resizebox{\textwidth}{!}{
    \begin{tabular}{|c|c|c|}
        \hline
        \textbf{Bestand} & \textbf{Graad} & \textbf{Motivatie} \\[0.5ex]
        \hline
        Een python functie met één functie: \ref{bijlage:makkelijk} & makkelijk & Eén functie \\
        \hline
        Een python bestand met verschillende functies: \ref{bijlage:gemiddeld} & gemiddeld &  Meerdere functies \\
        \hline
        Een python bestand met functies en klassen: \ref{bijlage:moeilijk} & moeilijk & Functies en klassen \\
        \hline
        Een complex python bestand met verschillende functies en klassen en nested functies: \ref{bijlage:extreem-moeilijk} & Extreem moeilijk & Nested functies en klassen \\
        \hline
    \end{tabular}}
    \caption{Aantal functies en klassen in de verschillende Python bestanden.}
    \label{table:bestanden}
\end{table}

In deze versie wordt er door de AST gelopen en wordt er gekeken of de node een functie of klasse is met de juiste naam. 
Als dit het geval is, wordt de code van de functie of klasse vervangen door de gegenereerde docstring.
Het toevoegen van de docstring wordt gedaan door de nieuwe code van de functie of klasse in de AST te plaatsen op de plaats van de oude code.
En dan de nieuwe AST te gebruiken als de nieuwe code van het bestand.

\begin{listing}
    \caption[Code voor het vervangen van een docstring]{Vervangen van de code van een functie door de gegenereerde docstring. \ref{bijlage:vervangen-v1}}
    \label{lst:vervangen-v1}
    \begin{minted}{python}
    def replace_functions(self, functions):
        tree = self.tree
        for node in ast.walk(tree):
            if isinstance(node, (ast.FunctionDef, ast.AsyncFunctionDef)) and node.name in functions:
                new_func_def = ast.parse(functions[node.name]).body
                tree.body.insert(tree.body.index(node), new_func_def)        
        self.tree = tree

    \end{minted}
\end{listing}

Deze code werkte niet volledig zoals verwacht. De oude code werd niet verwijderd uit de AST zoals te zien in \ref{lst:uitkomst-vervangen-dubbel}. Dit zorgde voor dubbele functies en klassen in de AST.
Waarvan één met docstring en één zonder. Dit werd opgelost door de oude code te verwijderen uit de AST \ref{bijlage:vervangen-v2}.
Alsook werd de code aangepast zodat functies die binnen in een klasse gedefinieerd zijn ook vervangen kunnen worden.
Het verwijderen uit de lijst met functies werd ook toegevoegd zodat de functies die al vervangen zijn niet opnieuw vervangen worden.

\begin{listing}
    \caption{Stuk uit uitkomst van het vervangen van de code van een functie \ref{bijlage:uitkomst-gemiddeld}.}
    \label{lst:uitkomst-vervangen-dubbel}
    \begin{minted}{python}
    def crop_image(img, x1, y1, x2, y2):

        def crop_image(img: ndarray, x1: int, y1: int, x2: int, y2: int) -> ndarray:
            """
            Crop the input image to the specified dimensions.

            Args:
                img (ndarray): The input image.
                x1 (int): The starting x-coordinate for cropping.
                y1 (int): The starting y-coordinate for cropping.
                x2 (int): The ending x-coordinate for cropping.
                y2 (int): The ending y-coordinate for cropping.

            Returns:
                ndarray: The cropped image.
            """
            if x1 < 0 or y1 < 0 or x2 > img.shape[1] or (y2 > img.shape[0]):
                img, x1, x2, y1, y2 = pad_img_to_fit_bbox(img, x1, x2, y1, y2)
            return img[y1:y2, x1:x2, :]
    \end{minted}
\end{listing}

Deze versie werkte zoals verwacht voor bestanden met moeilijkheidsgraad makkelijk tot moeilijk, bestanden zonder ingewikkelde structuren zoals nested functies of nested klassen.
De code liep vast bij bestanden met extreem moeilijke moeilijkheidsgraad.
Hierdoor moest de code opnieuw aangepast worden omdat de gegenereerde docstrings niet altijd correct toegevoegd werden.
Een nadeel van het werken met AST is dat de parent node van nested functies niet opgeslagen worden.
Dit werd opgelost door het vervangen recursief te laten gebeuren, een betere oplossing dan het gebruiken van if else statements \ref{lst:vervangen-v3}.

\begin{listing}
    \caption[Code voor het vervangen van een docstring v2]{Vervangen van de code van een functie door de gegenereerde docstring. \ref{bijlage:vervangen-v3}}
    \label{lst:vervangen-v3}
    \begin{minted}{python}
    def _replace_functions(self, node, functions):
        if isinstance(node, (ast.FunctionDef, ast.AsyncFunctionDef)) and node.name in functions:
            new_func_def = ast.parse(functions[node.name]).body[0]
            new_func_def.body.extend(node.body)
            parent_node = self._get_parent_node(node)
            index = parent_node.body.index(node)
            parent_node.body.remove(node)
            parent_node.body.insert(index, new_func_def)
            functions.pop(node.name)
        for child_node in ast.iter_child_nodes(node):
            self._replace_functions(child_node, functions)
    \end{minted}
\end{listing}

Door de code recursief te laten lopen, kan de code van nested functies en klassen ook vervangen worden.
Door het gebruiken van de parent node van de node die vervangen dient te worden, kan de docstring op de juiste index geplaatst worden.
De nieuwe node wordt toegevoegd aan de parent node en de oude node wordt verwijderd.
Zo kunnen grote Python bestanden met complexe structuren ook correct vervangen worden.

\subsection{Bestand samenvatting genereren}
\label{sec:bestanddocumentatie-samenvatting}
De laatste stap in het proces van het documenteren van een Python bestand is het genereren van een samenvatting van het bestand.
Deze samenvatting wordt gemaakt op basis van de reeds gegenereerde docstrings van de verschillende functies en klassen van het bestand.
Het gebruiken van een prompt waar alle docstrings meegegeven worden kan een correcte samenvatting als eindresultaat bekomen.
Hierin hoort er een korte beschrijving van het bestand te staan en een lijst van de functies en klassen die in het bestand voorkomen.
Er komt een beschrijving in de samenvatting van de functies en klassen die in het bestand voorkomen.
Deze beschrijvingen worden gegenereerd door het model op basis van de gegenereerde docstrings.

Deze samenvatting wordt gegenereerd door het model de gegenereerde docstrings van de functies en klassen mee te geven in een prompt, zoals de prompt \ref{bijlage:bestand-samenvatting}.
Dit is de laatste stap in het proces van het documenteren van een Python bestand.

Deze samenvatting kan dan gebruikt worden als basis voor het genereren van de verdere documentatie voor een Python project.
Een samenvatting op project niveau en de relaties tussen de verschillende bestanden van het project.

% \section{LLM voor documentatie}
% \label{sec:llm-voor-documentatie}

% Nadat de basiskennis over LLM's gelegd is, hoe deze werken en wat ze doen. Wat enkele bekende LLM's zijn en wat hun mogelijkheden zijn. 
% Ook zijn er bestaande tools besproken die documentatie genereren voor projecten.

% Omdat dit onderzoek gaat over het documenteren van een Python project met behulp van LLM's, moet er gekeken worden naar hoe LLM's gebruikt kunnen worden voor het genereren van documentatie.
% Een LLM kan gebruikt worden voor de verschillende documentatie stukken, samenvattingen, docstrings, relatie tussen de verschillende delen van het project, \dots
% Door de verschillende delen van het project meegegeven aan de LLM met een uitgebreid prompt. 
% Hier kan er telkens aan de Large Language Model gevraagd worden om een samenvatting te maken van wat dit deel van het project doet en wat de uitkomst is.
% Door dit te herhalen voor alle files van het project kan er één samenvattend document gemaakt worden van het gehele project.

% De LLM kan de relaties tussen de verschillende delen van het project leren en vastnemen.
% Zo wordt er een duidelijk beeld gevormd van de structuur van het project en wat de samenhang is van de verschillende delen van het project.
% Alle functies van het python project kunnen hier makkelijk teruggevonden worden.

% Er kan ook gebruik gemaakt worden van de LLM om de docstrings van de verschillende functies en klassen te genereren. 
% Om zo een betere samenvatting te verkrijgen van de werking van de verschillende delen van het project, door de docstrings te combineren met de samenvattingen van de LLM.
% Door telkens de docstrings en de naam van de verschillende functies en klassen mee te geven aan de LLM kan er een betere samenvatting gemaakt worden van het gehele project.
% =========================
% Project Documentation
% =========================

\chapter{\IfLanguageName{dutch}{Project documentatie}{Project documentation}}
\label{ch:project-documentatie}

\section{Inleiding}
\label{sec:project-documentatie-inleiding}

In dit hoofdstuk wordt er gekeken naar hoe de individuele samenvattingen van een Python bestand gebruikt kunnen worden om een samenvatting van het gehele project te maken.
Alsook wordt er verder gekeken naar hoe de relaties tussen de verschillende bestanden gevisualiseerd kunnen worden.
Dit om een zo goed mogelijk overzicht te krijgen van het project, zonder dat er handmatig documentatie moet worden geschreven.

\section{Project Samenvatting}
\label{sec:project-documentatie-samenvatting}

De samenvatting van een Python project kan gemaakt worden door de individuele samenvattingen van de bestanden samen te voegen.
Deze samenvatting kan bekomen worden door elk python bestand in het project te laten documenteren en de samenvatting ervan op te slaan.
Erna kunnen deze samenvattingen samengevoegd worden om dan mee te geven aan een Large Language Model.
Deze samenvattingen worden gegenereerd door het aanroepen van de functie \mintinline{python3}|generate_file_summaries()|. 
Deze functie maakt gebruik van de klasse \mintinline{python3}|FileDocumenationGenerator()| die de samenvattingen van de bestanden genereerd en opslaat in een dictionary.
De code van deze functie is te vinden in \ref{bijlage:file-summary-functions}.  

Door een duidelijk prompt mee te geven aan het model, kan er specifiek gevraagd worden welke functies en klasses er in het project zitten.
En dat per bestand duidelijk opgelijst. Samen met deze oplijsting wordt er ook een korte samenvatting van het gehele project weer gegeven.

Een voorbeeld van de uitkomst is te zien in \ref{fig:project-summary}.
Hier is te zien dat er een duidelijk overzicht is van welke functies en klasses er in het project zitten.
Alsook wat het gehele project inhoudt.

\begin{figure}[h]
    \centering
    \includegraphics[width=1\textwidth]{project_summary.png}
    \caption{Voorbeeld van een project samenvatting}
    \label{fig:project-summary}
\end{figure}    

\subsection{Keuze van welke bestanden te documenteren}
\label{subsec:project-documentatie-keuze-bestanden}

Het is belangrijk om te kijken naar welke bestanden er gedocumenteerd moeten worden.
Omdat het gaat over een Python project, is het belangrijk dat alle Python bestanden gedocumenteerd worden.
Er is de keuze gemaakt om het bestand \mintinline{python3}|__init__.py| niet te documenteren, omdat dit bestand vaak niet relevant is voor de documentatie van het project.
Dit omdat het bestand vaak leeg is of slechts minimale functionaliteit bevat. 
Ook bevat het soms enkele configuratie opties die niet relevant zijn voor de documentatie.

\subsection{Documentatie van bestanden zonder functies of klasses}
\label{subsec:project-documentatie-geen-functies}

Omdat er eerst vanuit gegaan wordt dat elk bestand functies of klasses bevat, is het belangrijk om te kijken naar bestanden die dit niet bevatten.
Deze bestanden dienen ook gedocumenteerd te worden om een volledig overzicht te krijgen van het project.
Als er geen apart prompt voorzien wordt dan zal het model hallucineren en een samenvatting verzinnen, dit is niet de bedoeling.
Er is gebruik gemaakt van een prompt \ref{bijlage:bestand-zonder-functies} die vraagt om de werking van het document uit te leggen en de eventuele imports die het bestand bevat.
In dit prompt wordt er duidelijk gedefinieerd wat er in de documentatie moet staan. 
En aan de hand van een voorbeeld wordt er getoond hoe de documentatie eruit moet zien.
Een voorbeeld van de uitkomst in de project documentatie van een bestand zonder functies is te zien in \ref{fig:file-no-functions}.

\begin{figure}[h]
    \centering
    \includegraphics[width=1\textwidth]{documentatie_bestand_zonder_functies.png}
    \caption{Voorbeeld van de documentatie van een bestand zonder functies of klasses}
    \label{fig:file-no-functions}
\end{figure}

\section{Prompting voor project documentatie}
\label{sec:project-documentatie-prompting}

Aangezien de documentatie van een project bestaat uit de documentatie van individuele Python bestanden, is het belangrijk dat deze op een correcte manier gedocumenteerd worden.
Dit wordt gerealiseerd met behulp van een prompt die de samenvatting van een Python bestand op een correcte manier interpreteert en omzet naar de juiste documentatie.

De gehele samenvatting van het project werd gemaakt door de individuele samenvattingen van de bestanden samen te voegen. 
En dit mee te geven met het prompt \ref{bijlage:prompt6}.
Dit prompt vraagt om de samenvatting van het project te maken en uit de individuele samenvattingen de functies en klasses op te lijsten.
Dit prompt werkte goed voor kleine Python projecten, omdat er bij de gekozen LLM een beperkt context window van 16,385 tokens \textcite{OpenAI}.

Hierdoor was het niet mogelijk om de gewenste samenvatting te creeren door alle individuele samenvattingen mee te geven aan het model.
Een oplossing die gevonden werd is de volgende: er werden verschillende kleinere prompts meegegeven met het model.
Dit prompt maakt per samenvatting van een Python bestand de documentatie van de verschillende functies en klasses. 
De functies en klasses worden opgelijst met het juiste formaat en een kleine uitleg.
Deze resultaten van de verschillende kleine prompts werden dan code matig samengevoegd tot een geheel. 
Het prompt dat gebruikt werd is te zien in \ref{bijlage:prompt7}.
Dit prompt vraagt om de functies en klasses van een Python bestand op te lijsten en een korte uitleg te geven van wat deze functies en klasses doen.
De uitkomst is weergegeven met een duidelijk voorbeeld binnen het prompt en volgens een markdown formaat. 
De documentatie genereerd door dit prompt is te zien in \ref{fig:project-summary}.

\section{Visualisatie van relaties tussen bestanden}
\label{sec:project-documentatie-relaties}

Om een goed overzicht te krijgen van het project is het belangrijk om de relaties tussen de verschillende bestanden te visualiseren.
Dit kan gedaan worden door gebruik te maken van graven om de relaties tussen de bestanden weer te geven.

Door gebruikt te maken van de tool \textcite{WHIR2018} een soort gelijke tool als degene die gebruikt wordt door \textcite{Doxygen2023}.
Met deze tool kunnen er graven gemaakt worden om zo de relaties tussen de verschillende bestanden weer te geven.
Deze graven kunnen dan gebruikt worden om een duidelijk overzicht te krijgen van het project.

\subsection{Genereren van de relaties tussen bestanden in een project}
\label{subsec:project-documentatie-relaties-genereren}

Om de relaties te bekomen tussen alle bestanden en mappen in een project is er een prompt meegegeven aan het Large Language Model.
In dit prompt worden de imports van alle bestanden opgelijst en wordt er gevraagd om de relaties tussen de bestanden weer te geven.
Doordat er telkens wanneer een bestand een andere functie of klasse van een ander bestand importeert staat dit in de imports van het bestand.
Hierdoor kan er gekeken worden naar de imports van de verschillende bestanden en zo de relaties tussen de bestanden visualiseren.

Het prompt dat gebruikt werd is te zien in \ref{bijlage:generate-file-relations}.
Er zijn verschillende iteraties van dit prompt gemaakt om de beste resultaten te bekomen.
De uitkomst van dit prompt is een CSV bestand met daarin het pad van het bestand, de bestandsnaam, het pad van de folder waarin het bestand zit en een lijst van alle geimporteerde bestanden.
Doordat GPT3.5 \autocite{OpenAI} een zo goed mogelijk antwoord probeert te geven op de vraag, is het belangrijk om de vraag zo duidelijk mogelijk te stellen.
Door verschillende voorbeelden mee te geven in het prompt kan het model een beter antwoord geven.
Omdat het voorbeeld van cruciaal belang is, heeft het enige tijd geduurd om het juiste prompt te vinden en de kinderziektes eruit te halen.
Zo zijn schrijffouten en onduidelijkheden in het prompt aangepast om een beter resultaat te bekomen.

\subsection{Visualisatie van de relaties}
\label{subsec:project-documentatie-relaties-visualisatie}

Eens er een goed CSV bestand is bekomen kunnen de relaties tussen de bestanden gevisualiseerd worden.
Dit wordt gedaan met behulp van de tool \textcite{WHIR2018}.
Deze tool haalt de relaties uit het CSV bestand en voegt deze toe aan een graaf. 
Dit door eerst de verschillende nodes toe te voegen en dan de edges tussen de nodes. 
De code waarop dit gebeurt is te vinden in \ref{bijlage:graph-functions}.
Als dit voor alle bestanden gedaan is, kan er een duidelijk overzicht bekomen worden van de relaties tussen de bestanden door de graaf te exporteren naar een HTML bestand.
Dit HTML bestand kan dan geopend worden in een browser om de graaf te bekijken, alsook kunnen de nodes versleept worden om een beter overzicht te bekomen.

\begin{figure}[h]
    \centering
    \includegraphics[width=1\textwidth]{graph.png}
    \caption{Voorbeeld van een graaf van de relaties tussen bestanden}
    \label{fig:graph}
\end{figure}

Er is een voorbeeld van een graaf te zien in \ref{fig:graph}.
Ook is er voor een groot project een graaf gemaakt om te kijken of de relaties tussen de bestanden duidelijk weergegeven worden \ref{fig:graph-large}.
De relaties op deze graaf zijn zichtbaar echter is het niet altijd duidelijk omdat er veel bestanden zijn en er bepaalde bestanden vaak geïmporteerd zijn.
Dit kan ervoor zorgen dat de graaf onoverzichtelijk wordt.

\begin{figure}[h]
    \centering
    \includegraphics[width=1\textwidth]{graph-large.png}
    \caption{Voorbeeld van een fragment van een graaf van de relaties tussen bestanden van een groot project}
    \label{fig:graph-large}
\end{figure}
%\input{...}
%...

%%=============================================================================
%% Conclusie
%%=============================================================================

\chapter{Conclusie}%
\label{ch:conclusie}

% TODO: Trek een duidelijke conclusie, in de vorm van een antwoord op de
% onderzoeksvra(a)g(en). Wat was jouw bijdrage aan het onderzoeksdomein en
% hoe biedt dit meerwaarde aan het vakgebied/doelgroep? 
% Reflecteer kritisch over het resultaat. In Engelse teksten wordt deze sectie
% ``Discussion'' genoemd. Had je deze uitkomst verwacht? Zijn er zaken die nog
% niet duidelijk zijn?
% Heeft het onderzoek geleid tot nieuwe vragen die uitnodigen tot verder 
%onderzoek?





% =============================

Aangezien dit onderzoek een beperkte scope heeft, zijn er enkele uitbreidingen die kunnen worden toegevoegd om het onderzoek te verbeteren.
Deze uitbreidingen kunnen helpen om de resultaten van het onderzoek te verbeteren en om de tool verder te ontwikkelen.

Zo kan er gekeken worden naar het genereren van documentatie voor andere programmeertalen.
Deze bachelorproef focust zich op Python, maar het is mogelijk om de tool uit te breiden naar andere programmeertalen.
Aangezien een Large Language Model zoals GPT \autocite{OpenAi2024} ook getraind zijn op andere programmeertalen.

Ook kan er gekeken worden naar hoe projecten met syntax fouten of andere problemen gedocumenteerd kunnen worden.
Dit is belangrijk omdat de tool nu enkel werkt op projecten die correcte syntax hebben.
Deze fouten kunnen eruit gehaald worden door de code eerst door een linter te halen en dan pas de documentatie te genereren.
De bekomen syntax fouten kunnen dan meegegeven worden aan een model om zo een bestand te genereren zonder syntax fouten.

Een andere uitbreiding is kijken naar hoe de documentatie geëvalueerd kan worden.
Omdat dit nu slechts manueel gebeurt, op basis van gezond verstand. 
Er kan gekozen worden om enquêtes af te nemen bij programmeurs om zo de documentatie te evalueren.
De evaluatie van de respondenten gaat echter slechts relatief zijn, omdat de respondenten beoordelen op basis van kennis van de programmeertaal. 
Of er kan gekeken worden naar hoe de documentatie van de tool vergeleken kan worden met de documentatie van de programmeur zelf.
Hier is het belangrijk om te kijken naar de verschillen en overeenkomsten tussen de documentatie van de tool en de documentatie van de programmeur.

Een laatste voorbeeld van een uitbreiding is om te kijken naar hoe verschillende Large Language Models presteren op het genereren van documentatie.
Zo kan er gekozen worden tussen modellen zoals GPT-4 \autocite{OpenAI2023}, LLama 2 \autocite{Meta2024}, Gemini \autocite{Google2024}, \dots

Sommige modellen hebben een groter context window dan andere modellen, zo zou er meer informatie meegegeven kunnen worden aan het model.
En het zou mogelijk een beter resultaat kunnen geven.

%---------- Bijlagen -----------------------------------------------------------

\appendix

\chapter{Onderzoeksvoorstel}

Het onderwerp van deze bachelorproef is gebaseerd op een onderzoeksvoorstel dat vooraf werd beoordeeld door de promotor. Dat voorstel is opgenomen in deze bijlage.

%% TODO: 
\section{Samenvatting}

% Kopieer en plak hier de samenvatting (abstract) van je onderzoeksvoorstel.
\begin{abstract}
    Documentatie van een Python project is belangrijk, maar het is een tijdrovende taak en het wordt vaak niet grondig gedaan.
    Deze bachelorproef aan de HoGent onderzoekt het automatisch genereren van documentatie voor python projecten met behulp van Large Language Modellen.
    Er wordt een tool ontwikkeld die de Python code en de relaties tussen de verschillende bestanden analyseert en op basis daarvan een overzichtelijke documentatie genereerd.
    Er wordt gekeken naar hoe de documentatie van Python functies gebruikt kunnen worden voor het maken van een gehele samenvatting van het project.
    Dit wordt gedaan op basis van huidige methoden om docstrings aan te maken en te gebruiken.
    Deze informatie kan dan gegeven worden aan de Large Language Modellen om een samenvatting te genereren.
    
    Er worden verschillende Python projecten verzameld en geanalyseerd om te kijken hoe de documentatie gegenereerd kan worden.
    Dan worden er LLMs getraind op basis van deze projecten en wordt er gekeken naar hoe de documentatie gegenereerd kan worden.
    De gegenereerde documentatie kan dan vergeleken worden met de huidige documentatie van de projecten om dit te evalueren.
    Ook zal er gevraagd worden aan enkele programmeurs om de documentatie te evalueren.
    
    Op basis van deze feedback kan het model gefinetuned worden. Er kan gekeken worden naar de mogelijke verbeteringspunten zodat er uiteindelijk een betere documentatie van het project ontstaat.
    Het resultaat is dat er een tool is die de documentatie van een Python project kan genereren.
    Dit resultaat maakt het mogelijk om de gegenereerde samenvatting van een Python project te lezen. 
    De lezer kan dan stukken gebruiken uit het project of er verder mee aan de slag gaan.
\end{abstract}

% Verwijzing naar het bestand met de inhoud van het onderzoeksvoorstel
%---------- Inleiding ---------------------------------------------------------

\section{Introductie}%
\label{sec:introductie}

Documentatie is belangrijk wanneer er aanpassingen moeten gebeuren aan de code van een project. Ook moet een iemand anders de code kunnen begrijpen voordat de code gebruikt kan worden binnen een ander project.
Hoe kan geautomatiseerde documentatiegeneratie met behulp van Large Language Modellen (LLM) effectief worden toegepast om een duidelijk overzichten en informatieve beschrijvingen te produceren voor Python projecten?

Het is belangrijk dat de skill of de knowhow van een project gedeeld kan worden met anderen. 
Door het toepassen van documentatie kan deze kennis makkelijk vergaard worden door andere geïnteresseerden.
Het is dus een belangrijk dat er aan documentatie gedaan wordt en dat deze up-to-date blijft. 

Documentatie is iets dat veel tijd kost, dat vaak niet gemaakt wordt en het is iets dat up-to-date gehouden moet worden.
Het gebruiken ervan kan ervoor zorgen dat er geen dubbel werk gedaan moet worden. Een tool die dit proces kan versnellen / automatiseren zou een grote meerwaarde zijn.
De tool bestaat uit een geautomatiseerde documentatie LLM die de project code analyseert en samenvat in een document. 
Dit geeft de werknemers de moegelijkheid om zich in te lezen in het project en erna zelf aanpassingen te maken of stukken code te gebruiken voor een ander project.

Het eindresultaat van deze bachelorproef is een Proof of Concept (PoC) van een geautomatiseerde tool die de project code analyseert en er documentatie van genereert.
De gegenereerde documentatie laat het toe om het project te begrijpen zonder er te veel tijd aan te besteden.

%---------- Stand van zaken ---------------------------------------------------

\section{Literatuurstudie}%
\label{sec:Literatuurstudie}

Wat is documentatie binnen Python projecten en wat zijn de huidige tools?
Voor de taal Python bestaan er al verschillende tools die documentatie genereren voor blokken code zoals pdoc \autocite{GallantHils2023} en Sphinx \autocite{Sphinx2023}. 
Met behulp van de sphinx autodoc functie \autocite{Sphinx2023} kan een python functie omschreven worden in een docstring.
Een docstring is een blok tekst dat de werking van een python functie omschrijft. Door deze beknopte blok tekst wordt er duidelijk wat de functie doet.
Deze docstrings kunnen mogelijks gebruikt worden bij het maken van een document dat het project omschrijft.

Er is ook al onderzoek gedaan naar het automatisch genereren van documentatie van code blokken met behulp van een Neural Attention Model (NAM) \autocite{IyerEtAl2016}.
Dit onderzoek heeft gekeken naar het genereren van hoogstaande samenvattingen van source code. 
Het maakt gebruik van neurale netwerken die stukken C\# code en SQL queries omzetten naar zinnen die de code omschrijven. 
Dit helpt bij het begrijpen van stukken code maar niet van een geheel project waar meerdere bestanden bij betrokken zijn.

Hoe kunnen Large Language Modellen gebruikt worden om documentatie te genereren?
Large Language Modellen zijn neurale netwerken die getraind worden op grote hoeveelheden tekst. 
Deze modellen kunnen tekst genereren op basis van een gegeven input. De mogelijkse invoer in deze bachelorproef kan dan een python project zijn, of de docstrings van verschillende functies in een project.
Er wordt dan verwacht dat het een samenvatting maakt van het project of van de functies.

Ook kan er gekeken worden naar GitHub README.md bestanden. Dit zijn bestanden waarin de werking van een project kort wordt uitgelegd. Deze zijn echter niet altijd makkelijk te lezen. 
Volgens de studie \textcite{GaoEtAl2023} kan de tekst vereenvoudigd worden terwijl steeds de correcte betekenis te behouden, en dit aan de hand van een transfer learning model.
Het gebruik van LLMs en het automatisch code documentatie met behulp van syntax bomen wordt onderzocht in \textcite{Procko2023} voor de C\# en .NET programeertalen.
In deze studie wordt er gekeken naar het gebruik van LLMs specifiek GPT-3.5 en GPT-4 om code te documenteren.

In de studie van \textcite{McBurneyMcMillan2014} wordt onderzocht hoe er automatisch documentatie gegenereerd kan worden voor Java code, specifiek naar hoe de methodes met elkaar verbonden zijn en welke rol ze spelen binnen het project.
Dit is een mooi voorbeeld van wat ik wil bereiken met deze bachelorproef, een samenhangend geheel van verschillende bestanden van een python project die samen een duidelijk overzicht geven van de werking van het project.
Hoe kunnen LLMs gebruikt worden om automatisch documentatie te genereren voor python projecten.
De gegenereerde samenvatting van het gehele project geeft de lezer ervan de mogelijkheid het project te gebruiken of aan te passen zonde de totale project code te ontleden.

%---------- Methodologie ------------------------------------------------------
\section{Methodologie}%
\label{sec:methodologie}

Hier beschrijf je hoe je van plan bent het onderzoek te voeren. Welke onderzoekstechniek ga je toepassen om elk van je onderzoeksvragen te beantwoorden? 
Gebruik je hiervoor literatuurstudie, interviews met belanghebbenden (bv.~voor requirements-analyse), experimenten, simulaties, vergelijkende studie, risico-analyse, PoC, \ldots?

Valt je onderwerp onder één van de typische soorten bachelorproeven die besproken zijn in de lessen Research Methods (bv.\ vergelijkende studie of risico-analyse)? 
Zorg er dan ook voor dat we duidelijk de verschillende stappen terug vinden die we verwachten in dit soort onderzoek!

Vermijd onderzoekstechnieken die geen objectieve, meetbare resultaten kunnen opleveren. 
Enquêtes, bijvoorbeeld, zijn voor een bachelorproef informatica meestal \textbf{niet geschikt}. 
De antwoorden zijn eerder meningen dan feiten en in de praktijk blijkt het ook bijzonder moeilijk om voldoende respondenten te vinden. 
Studenten die een enquête willen voeren, hebben meestal ook geen goede definitie van de populatie, 
waardoor ook niet kan aangetoond worden dat eventuele resultaten representatief zijn.

Uit dit onderdeel moet duidelijk naar voor komen dat je bachelorproef ook technisch voldoen\-de diepgang zal bevatten. 
Het zou niet kloppen als een bachelorproef informatica ook door bv.\ een student marketing zou kunnen uitgevoerd worden.

Je beschrijft ook al welke tools (hardware, software, diensten, \ldots) je denkt hiervoor te gebruiken of te ontwikkelen.

Probeer ook een tijdschatting te maken. Hoe lang zal je met elke fase van je onderzoek bezig zijn en wat zijn de concrete \emph{deliverables} in elke fase?

%---------- Verwachte resultaten ----------------------------------------------
\section{Verwacht resultaat, conclusie}%
\label{sec:verwachte_resultaten}

Hier beschrijf je welke resultaten je verwacht. Als je metingen en simulaties uitvoert, kan je hier al mock-ups maken van de grafieken samen met de verwachte conclusies. 
Benoem zeker al je assen en de onderdelen van de grafiek die je gaat gebruiken. 
Dit zorgt ervoor dat je concreet weet welk soort data je moet verzamelen en hoe je die moet meten.

Wat heeft de doelgroep van je onderzoek aan het resultaat? Op welke manier zorgt jouw bachelorproef voor een meerwaarde?

Hier beschrijf je wat je verwacht uit je onderzoek, met de motivatie waarom. Het is \textbf{niet} erg indien uit je onderzoek andere 
resultaten en conclusies vloeien dan dat je hier beschrijft: het is dan juist interessant om te 
onderzoeken waarom jouw hypothesen niet overeenkomen met de resultaten.



%%---------- Andere bijlagen --------------------------------------------------
% TODO: Voeg hier eventuele andere bijlagen toe. Bv. als je deze BP voor de
% tweede keer indient, een overzicht van de verbeteringen t.o.v. het origineel.
% ==========================================
% Bijlage
% ==========================================

\chapter{Bijlage}
\label{bijlage}

In deze bijlage worden de volledige code van de applicatie en de gebruikte prompts weergegeven.

\section{Prompts}
\label{bijlage:prompts}

\subsection{Function Prompt 1}
\label{bijlage:prompt1}
Instructies voor het genereren van een docstring voor een functie versie 1.
\begin{minted}{python}
    '''For this Python function:
    ```python	
    def is_prime(n):
    if n in [2, 3]:
        return True
    if (n == 1) or (n % 2 == 0):
        return False
    r = 3
    while r * r <= n:
        if n % r == 0:
            return False
        r += 2
    return True
    ```
    Leave out any imports, just return the function with the docstring and type hints.
    The function, with docstring using the google docstring style and with type hints is:
    ```python	
    def is_prime(n: int) -> bool:
    """
    Check if a number is prime.
    Args:
        n (int): The number to check.
    Returns:
        bool: True if the number is prime, False otherwise.
    """
    if n in [2, 3]:
        return True
    if (n == 1) or (n % 2 == 0):
        return False
    r = 3
    while r * r <= n:
        if n % r == 0:
            return False
        r += 2
    return True
    ```
    
    For this Python function:
    ```python	
    {code}
    '''
\end{minted}

\subsection{Function Prompt 2}
\label{bijlage:prompt2}
Instructies voor het genereren van een docstring voor een functie versie 2.
\begin{minted}{python}
'''
    The following Python function is a code snippit from a Python file. 
    The following function lacs a docstring and type hints.
    Your task is to add a docstring and type hints to the function.
    You can't change the function's code, add any imports, or assume anything about the function's behavior or datatypes that is not clear from the code snippet itself.
    Below is a function that needs a docstring and type hints:
    ```python	
    def is_prime(n):
    if n in [2, 3]:
        return True
    if (n == 1) or (n % 2 == 0):
        return False
    r = 3
    while r * r <= n:
        if n % r == 0:
            return False
        r += 2
    return True
    ```
    The correct outcome should be the following Python code:
    ```python	
    def is_prime(n: int) -> bool:
    """
    Check if a number is prime.
    Args:
        n (int): The number to check.
    Returns:
        bool: True if the number is prime, False otherwise.
    """
    if n in [2, 3]:
        return True
    if (n == 1) or (n % 2 == 0):
        return False
    r = 3
    while r * r <= n:
        if n % r == 0:
            return False
        r += 2
    return True
    ```
    
    Now it's your turn to add a docstring and type hints to the following function:
    ```python	
    {code}
    ```
    '''
\end{minted}


\subsection{Function Prompt 3}
\label{bijlage:prompt3}
Prompt versie 3 voor het genereren van een docstring voor een functie. 
\begin{minted}{python}
    '''You are an AI documentation assistant, and your task is to generate docstrings and typehints based on the given code of a function, the function is a code snippet from a Python file.
    Do your task with the least amount of assumptions, you can't add any imports, change the code, or assume anything about the function's behavior or datatypes that is not clear from the code snippet itself.
    The purpose of the documentation is to help developers and beginners understand the function and specific usage of the code.

    An example of your task is as follows:
    The given code is:

    ```python	
    def is_prime(n):
    if n in [2, 3]:
        return True
    if (n == 1) or (n % 2 == 0):
        return False
    r = 3
    while r * r <= n:
        if n % r == 0:
            return False
        r += 2
    return True
    ```

    The expected output of your task for the given code is:

    ```python	
    def is_prime(n: int) -> bool:
    """
    Check if a number is prime.

    Args:
        n (int): The number to check.

    Returns:
        bool: True if the number is prime, False otherwise.
    """

    if n in [2, 3]:
        return True
    if (n == 1) or (n % 2 == 0):
        return False
    r = 3
    while r * r <= n:
        if n % r == 0:
            return False
        r += 2
    return True
    ```

    Now it's your turn to generate the docstrings and typehints for the following function of a file with these imports:
    {imports}

    The content of the code is as follows:
    {code_content}
    '''
\end{minted}

\subsection{Class Prompt 1}
\label{bijlage:prompt4}
Prompt voor het genereren van een docstring voor een klasse.
\begin{minted}{python}
    '''
    You are an AI documentation assistant, and your task is to generate docstrings and typehints based on the given code of a class, the class is a code snippet from a Python file.
    Do your task with the least amount of assumptions, you can't add any imports, change the code, or assume anything about the classes behavior or datatypes that is not clear from the code snippet itself.
    The purpose of the documentation is to help developers and beginners understand the function and specific usage of the code.

    An example of your task is as follows:
    The given code is:

    ```python
    class Circle:
        def __init__(self, radius: float) -> None:
            """
            Initialize the Circle object with a given radius.

            Args:
                radius (float): The radius of the circle.
            """
            self.radius = radius

        def calculate_area(self) -> float:
            """
            Calculate the area of the circle.

            Returns:
                float: The area of the circle.
            """
            return round(math.pi * self.radius ** 2, 2)

        def calculate_circumference(self) -> float:
            """
            Calculate the circumference of the circle.

            Returns:
                float: The circumference of the circle.
            """
            return round(2 * math.pi * self.radius, 2)
    ```

    The expected output of your task for the given code is:

    ```python
    class Circle:
        """
        A class representing a circle with methods to calculate its area and circumference.

        Attributes:
            radius (float): The radius of the circle.

        Methods:
            __init__: Initialize the Circle object with a given radius.
            calculate_area: Calculate the area of the circle.
            calculate_circumference: Calculate the circumference of the circle.
        """
    ```

    Now it's your turn to generate the docstrings and typehints for the following class of a file with these imports:
    {imports}

    The content of the code is as follows:
    {code_content}

    Only generate the class docstring

    '''
\end{minted}

\subsection{Samenvatting van een bestand}
\label{bijlage:bestand-samenvatting}
Prompt voor het genereren van een samenvatting van een bestand.
\begin{minted}{python}
    '''
    You are an AI documentation assistant, and your task is to generate a summary of the given Python file. 
    The summary should include the following information:
    - What the file does.
    - What classes are defined in the file.
    - What functions are defined in the file.
    - And a brief description of each class and function.
    - Include the file name at the beginning of the summary.
    
    You are going to generate the summary based on given function names, class names and their docstrings.
    
    Now it's your turn to generate the summary given the following code of the file: {filename}:

    {code_content}
    '''
    \end{minted}

\subsection{Bestand zonder functies of klasses}
\label{bijlage:bestand-zonder-functies}
Prompt voor het genereren van een samenvatting van een bestand zonder functies of klasses.
\begin{minted}{python}
    '''
    You are an AI documentation assistant, and your task is to generate a summary of the given Python file based on the code content.
    The summary should include the following information:
    - What the file does.
    - What is the purpose of the file.
    - What is the main functionality of the file.
    - What the output is
    - What it does when executed.
    - Include the file name at the beginning of the summary.

    An example of the output of your task is as follows:
    Given the following code content:

    ```python
    from model import get_model
    from train import train_top_layer, train_all_layers
    if __name__ == '__main__':
        model = get_model()
        train_top_layer(model)
    ```

    The expected output of your task for the given code is the summary of the file:
    
    ```python
    """
    Summary of file: main.py
    
    This file contains the main functionality for a Python application.
    It imports the get_model function from the model module and the train_top_layer and train_all_layers functions from the train module.
    When executed, it gets a model using the get_model function and trains the top layer of the model using the train_top_layer function.
    """
    ```

    You are going to generate the summary based on the given code content of the file with filename: {filename}.
    {code_content}
    '''
\end{minted}

\subsection{Project samenvatting}
\label{bijlage:prompt6}
Prompt voor het genereren van een samenvatting van een project.
\begin{minted}{python}
    '''
    You are an AI documentation assistant, and your task is to generate a summary of the given Python project.
    The summary should include the following information:
    - What the project does.
    - What files are included in the project. And what each file does. What functions and classes are defined in each file.
    - A brief description of each class and function.
    - Include the project name at the beginning of the summary.

    You are going to generate the summary based on summaries of each file in the project.
    
    Now it's your turn to generate the summary given the following project structure:
    {project_name}

    With the following folder structure:
    {folder_structure}

    And the following summaries of each file:
    {summaries}
    '''
\end{minted}

\subsection{Project samenvatting per file}
\label{bijlage:prompt7}
Prompt voor het genereren van een samenvatting van een project per bestand.
\begin{minted}{python}
'''
    You are an AI documentation assistant, and your task is to generate a markdown summary of a file.
    For the following file summary:
    """
    Summary of file: crop_images.py

    This file contains the implementation of functions for cropping and padding images.

    Functions:
        crop_faces: Crop faces from an image using a specified bounding box.
        crop_image: Crop a specified region from an image.
        pad_img_to_fit_bbox: Pad an image to fit a specified bounding box.
    """

    The output should be:
    - **crop_images.py**: 
        - Contains functions for cropping and padding images:
          - `crop_faces`: Crop faces from an image using the given bounding boxes.
          - `crop_image`: Crop a specific region from an image based on the provided coordinates.
          - `pad_img_to_fit_bbox`: Pad an image to fit the specified bounding box.

    You are going to generate the markdown summary for the file: {file} with the following summary:
    {summary}
    '''
\end{minted}


\section{Code}
\label{bijlage:code}

\subsection{Bestand Documentatie}
\label{bijlage:documentatie-bestand}

\subsubsection{Vervangen van de code van een functie door de gegenereerde docstring. v1}
\label{bijlage:vervangen-v1}
\begin{minted}{python}
def replace_functions(self, functions):
    tree = self.tree
    for node in ast.walk(tree):
        if isinstance(node, (ast.FunctionDef, ast.AsyncFunctionDef)) and node.name in functions:
            new_func_def = ast.parse(functions[node.name]).body
            tree.body.insert(tree.body.index(node), new_func_def)        
    self.tree = tree
\end{minted}

\subsubsection{Vervangen van de code van een functie door de gegenereerde docstring. v2}
\label{bijlage:vervangen-v2}
Versie 2 van de functie om de code van een functie te vervangen door de gegenereerde docstring.
\begin{minted}{python}
def replace_functions(self, functions):
    tree = self.tree
    for node in ast.walk(tree):
        if isinstance(node, ast.ClassDef):
            for child_node in node.body:
                if isinstance(child_node, (ast.FunctionDef, ast.AsyncFunctionDef)) and child_node.name in functions:
                    new_func_def = ast.parse(functions[child_node.name]).body[0]
                    new_func_def.body.extend(child_node.body)
                    idx = node.body.index(child_node)
                    node.body.insert(idx, new_func_def)
                    node.body.remove(child_node)
                    functions.pop(child_node.name)
        elif isinstance(node, (ast.FunctionDef, ast.AsyncFunctionDef)) and node.name in functions:
            new_func_def = ast.parse(functions[node.name]).body[0]
            new_func_def.body.extend(node.body)
            tree.body.insert(tree.body.index(node), new_func_def)
            tree.body.remove(node)
            functions.pop(node.name)
    self.tree = tree
\end{minted}

\subsubsection{Vervangen van de code van een functie door de gegenereerde docstring. v3}
\label{bijlage:vervangen-v3}
\begin{minted}{python}
def _replace_functions(self, node, functions):
    if isinstance(node, (ast.FunctionDef, ast.AsyncFunctionDef)) and node.name in functions:
        new_func_def = ast.parse(functions[node.name]).body[0]
        new_func_def.body.extend(node.body)
        parent_node = self._get_parent_node(node)
        index = parent_node.body.index(node)
        parent_node.body.remove(node)
        parent_node.body.insert(index, new_func_def)
        functions.pop(node.name)
    for child_node in ast.iter_child_nodes(node):
        self._replace_functions(child_node, functions)
\end{minted}

\subsubsection{Genereren van de relaties tussen de verschillende bestanden}
\label{bijlage:generate-file-relations}

\begin{minted}{python}
'''
    You are an AI documentation assistant, and your task is to generate a csv file containing the relations between the files in a Python project.

    For the given project structure and imports:
    The imports are as follows:
    '1_RPS_Game\\CMD_version\\main.py': 'import random', '2_PyPassword_Generator\\CMD_version\\main.py': 'import random', '3_Hangman_Game\\CMD_version\\images.py': '', '3_Hangman_Game\\CMD_version\\main.py': 'import requests\nimport random\nimport os\nfrom images import hangman_logo\nfrom images import stages', '4_Hangman_Game\\CMD_version\\stages.py': '', '4_Hangman_Game\\CMD_version\\images.py': 'import csv\nimport matplotlib', '4_Hangman_Game\\CMD_version\\main.py': 'import random\nimport os\nfrom images import stages\nfrom images import logo'

    The structure of the project is as follows:
    '.': ['LICENSE', 'README.md'], '1_RPS_Game': [], '1_RPS_Game\\CMD_version': ['1_RPS_Game\\CMD_version\\main.py'], '2_PyPassword_Generator': [], '2_PyPassword_Generator\\CMD_version': ['2_PyPassword_Generator\\CMD_version\\main.py'], '3_Hangman_Game': [], '3_Hangman_Game\\CMD_version': ['3_Hangman_Game\\CMD_version\\images.py', '3_Hangman_Game\\CMD_version\\main.py'], '4_Hangman_Game': [], '4_Hangman_Game\\CMD_version': ['4_Hangman_Game\\CMD_version\\stages.py', '4_Hangman_Game\\CMD_version\\images.py', '4_Hangman_Game\\CMD_version\\main.py']

    The expected output of your task is the following:
    ```csv
File_Path,File_Name,Folder_Path,Uses_File
1_RPS_Game\CMD_version\main.py, main.py,1_RPS_Game\CMD_version,[]
2_PyPassword_Generator\CMD_version\main.py,main.py, 2_PyPassword_Generator\CMD_version,[]
3_Hangman_Game\CMD_version\images.py,images.py,3_Hangman_Game\CMD_version,[]
3_Hangman_Game\CMD_version\main.py, main.py,3_Hangman_Game\CMD_version,['3_Hangman_Game.CMD_version.images']
4_Hangman_Game\CMD_version\stages.py,stages.py,4_Hangman_Game\CMD_version,[]
4_Hangman_Game\CMD_version\images.py,images.py,4_Hangman_Game\CMD_version,[]
4_Hangman_Game\CMD_version\main.py,main.py,4_Hangman_Game\CMD_version,['4_Hangman_Game.CMD_version.images';'4_Hangman_Game.CMD_version.stages']
    ```

    The Column "Uses File" should only contain the files where the file imports functions from.
    For example if the imports are:
    ```python
    Import csv
    Import matplotlib
    from images import open_image
    from stages import stage1
    ```
    The Column "Uses File" should contain the file '4_Hangman_Game\\CMD_version\\images.py' and '4_Hangman_Game\\CMD_version\\stages.py'
    Do your task given the following imports and structure of the project:
    
    The imports are as follows:
    {imports}

    And the structure of the project is as follows:
    {structure}

    THE OUTPUT SHOULD BE A SINGLE CSV FILE CONTAINING THE RELATIONS BETWEEN THE FILES IN THE PROJECT.
    '''
\end{minted}

\subsection{Project Documentatie}

\subsection{Functies voor het samenvatting van een bestand}
\label{bijlage:file-summary-functions}
\begin{minted}{python}
    def document_file(self, file_path, outfolder_path):
        FDG = FileDocumenationGenerator(self.api_key, self.azure_endpoint, file_path, self.folder_path, outfolder_path)
        FDG.generate_file_documentation()
        return FDG

    def generate_file_summaries(self, python_files):
        for file in python_files:
            print("Documenting file: ", file)
            FDG = self.document_file(file, outfolder_path=self.outfolder)
            self.summaries[file] = FDG.get_summary()
            self.imports[file] = FDG.get_imports()
\end{minted}

\subsubsection{Generatie van een graph van de relaties tussen de bestanden}
\label{bijlage:generate-file-graph}

\begin{minted}{python}
def generate_graph_html(self):
        print("Generating graph html")
        added_edges = set()
        df = pd.read_csv(os.path.join(self.outfolder, 'graph_relations.csv'))
        net = Network(height="750px", width="100%", bgcolor="#222222", font_color="white") 
        net = Network(directed =True)
        net.add_node("root", shape='star', label="")     
        for index, row in df.iterrows():
            path = row['Folder_Path'].split("\\")
            # Add nodes for each folder in the path
            if len(path) > 1:
                for i in range(len(path)-1):
                    path_id = "_".join(path[:i+1])
                    net.add_node(path_id, label=path[i], shape='box')
                    next_path_id = "_".join(path[:i+2])
                    net.add_node("_".join(path[:i+2]), label=path[i+1], shape='box')                        
                    edge = (path_id, next_path_id)
                    if edge not in added_edges:
                        net.add_edge(path_id, next_path_id)
                        added_edges.add(edge)
            elif len(path) == 1:
                    net.add_node(path[0], label=path[0], shape='box')

            # Add node for the file
            file_path = row['File_Path'].split("\\")
            file_id = "_".join(file_path)
            parent_folder_id = "_".join(path)
            net.add_node(file_id, label=file_path[-1])
            edge = (parent_folder_id, file_id)
            if edge not in added_edges:
                net.add_edge(parent_folder_id, file_id)
                added_edges.add(edge)

            # Add edges for the root node
            root_edge = ("root", path[0])
            if root_edge not in added_edges:
                net.add_edge("root", path[0])
                added_edges.add(root_edge)
        
        for index, row in df.iterrows():
            file_id = "_".join(row['File_Path'].split("\\"))
            # Add edges for the uses files
            uses = row['Uses_File'].strip("[]")
            if uses:
                uses = uses.split(";")
                for use_file in uses:
                    use_file_path = use_file.strip("'").split(".")
                    use_file_id = "_".join(use_file_path)+".py"
                    edge = (file_id, use_file_id)
                    if edge not in added_edges:
                        net.add_edge(file_id, use_file_id)
                        added_edges.add(edge)

        path = os.path.join(self.outfolder, 'graph.html')
        net.save_graph(path)
\end{minted}


%%---------- Backmatter, referentielijst ---------------------------------------

\backmatter{}

\setlength\bibitemsep{2pt} %% Add Some space between the bibliograpy entries
\printbibliography[heading=bibintoc]

\end{document}
