% =========================
% Project Documentation
% =========================

\section{Project documentatie}
\label{sec:project-documentatie}

\subsection{Inleiding}
\label{sec:project-documentatie-inleiding}

In dit hoofdstuk wordt er gekeken hoe de individuele samenvattingen van een Python-bestand gebruikt kunnen worden om een samenvatting van het gehele project te maken.
Verder wordt onderzocht hoe de relaties tussen de verschillende bestanden gevisualiseerd kunnen worden met als doel een zo goed mogelijk overzicht van het project te krijgen zonder handmatige documentatie.

\subsection{Projectsamenvatting}
\label{sec:project-documentatie-samenvatting}

De samenvatting van een Python-project kan gemaakt worden door de individuele samenvattingen van de bestanden samen te voegen.
Deze samenvatting kan bekomen worden door elk Python-bestand in het project te laten documenteren en de samenvatting ervan op te slaan.
Daarna kunnen deze samenvattingen samengevoegd worden om dan mee te geven aan een Large Language Model.

Deze samenvattingen worden gegenereerd door het aanroepen van de functie \mintinline{python3}|generate_file_summaries()|. 
Deze functie maakt gebruik van de klasse \mintinline{python3}|FileDocumenationGenerator()| die de samenvattingen van de bestanden genereert en opslaat in een dictionary.
De code van deze functie is te vinden in \ref{bijlage:file-summary-functions}.  

Door een duidelijk prompt mee te geven aan het model kan er specifiek gevraagd worden welke functies en klassen er in het project zitten en dat duidelijk per bestand opgelijst. 
Samen met deze oplijsting wordt er ook een korte samenvatting van het gehele project weergegeven.
Een voorbeeld van de uitkomst is te zien in \ref{fig:project-summary}.
Hier is te zien dat er een duidelijk overzicht is van welke functies en klassen er in het project zitten, alsook wat het gehele project inhoudt.

\begin{figure}[h]
    \centering
    \includegraphics[width=1\textwidth]{project_summary.png}
    \caption{Voorbeeld van een project samenvatting}
    \label{fig:project-summary}
\end{figure}    

\subsubsection{Keuze van welke bestanden te documenteren}
\label{subsec:project-documentatie-keuze-bestanden}

Het is belangrijk om te kijken welke bestanden er gedocumenteerd moeten worden.
Omdat het gaat over een Python-project, is het belangrijk dat alle Python-bestanden gedocumenteerd worden.
Er is de keuze gemaakt om het bestand \mintinline{python3}|__init__.py| niet te documenteren, omdat dit bestand vaak niet relevant is voor de documentatie van het project.
Het bestand \mintinline{python3}|__init__.py| is vaak leeg of het heeft slechts een minimale functionaliteit. 

\subsubsection{Documentatie van bestanden zonder functies of klassen}
\label{subsec:project-documentatie-geen-functies}

Omdat er eerst vanuit gegaan wordt dat elk bestand functies of klassen bevat, is het belangrijk om te kijken naar bestanden die dit niet bevatten.
Deze bestanden dienen ook gedocumenteerd te worden om een volledig overzicht te krijgen van het project.
Als er geen apart prompt voorzien wordt, dan zal het model hallucineren en een samenvatting verzinnen. Dit is niet de bedoeling.
Er is gebruik gemaakt van een prompt \ref{bijlage:bestand-zonder-functies} die vraagt om de werking van het document uit te leggen en de eventuele imports die het bestand bevat.
In dit prompt wordt er duidelijk gedefinieerd wat er in de documentatie moet staan. 
Aan de hand van een voorbeeld wordt er getoond hoe de documentatie eruit moet zien.
Een voorbeeld van de uitkomst in de project documentatie van een bestand zonder functies is te zien in de figuur\ref{fig:file-no-functions}.

\begin{figure}[h]
    \centering
    \includegraphics[width=1\textwidth]{documentatie_bestand_zonder_functies.png}
    \caption{Voorbeeld van de documentatie van een bestand zonder functies of klassen}
    \label{fig:file-no-functions}
\end{figure}

\subsubsection{Prompting voor projectsamenvatting}
\label{sec:project-documentatie-prompting}

Aangezien de samenvatting van een project bestaat uit de samenvattingen van individuele Python-bestanden, is het belangrijk dat deze op een correcte manier gegenereerd worden.
Dit wordt gerealiseerd met behulp van een prompt die de samenvatting van een Python-bestand op een correcte manier interpreteert en omzet naar de juiste documentatie.

De gehele samenvatting van het project werd gemaakt door de individuele samenvattingen van de bestanden samen te voegen en dit mee te geven met het prompt \ref{bijlage:prompt6}.
De volgorde van de samenvattingen is niet van belang, omdat de relaties tussen de bestanden later nog gevisualiseerd worden in hoofdstuk~\ref{sec:project-documentatie-relaties} en dit op basis van de imports.
Dit prompt vraagt om de samenvatting van het project te maken en uit de individuele samenvattingen de functies en klassen op te lijsten.

Doordat het model beter kleine prompts kan verwerken, is er gekozen om de samenvattingen van de bestanden in kleinere stukken mee te geven aan het model.
Dit prompt maakt per samenvatting van een Python-bestand de documentatie van de verschillende functies en klassen. 
De functies en klassen worden opgelijst met het juiste formaat en een kleine uitleg.
Deze resultaten van de verschillende kleine prompts werden dan code matig samengevoegd tot een geheel. 
Door het code matig toevoegen van de individuele samenvattingen gaat er niets verloren. 
Deze keuze is verkozen omdat de LLM de uitkomst dan correct formateert in markdown.  
De uitkomst van de samenvatting van het project is te zien in de figuur \ref{fig:project-summary}.

Het prompt dat gebruikt werd, is te zien in \ref{bijlage:prompt7}.
Dit prompt vraagt om de functies en klassen van een Python-bestand op te lijsten en een korte uitleg te geven wat deze functies en klassen doen.
De uitkomst is weergegeven met een duidelijk voorbeeld binnen het prompt en volgens een markdown formaat. 
De documentatie gegenereerd door dit prompt is te zien in \ref{fig:project-summary}.

\subsection{Visualisatie van relaties tussen bestanden}
\label{sec:project-documentatie-relaties}

Om een goed overzicht te krijgen van het project is het belangrijk om de relaties tussen de verschillende bestanden te visualiseren.
Dit kan gedaan worden door gebruik te maken van graven om de relaties tussen de bestanden weer te geven.
Omdat er uit de shortlist is gebleken dat er een bestaande tool is die dit kan, namelijk \textcite{Doxygen2023}, is deze tool grondig bekeken.
\textcite{Doxygen2023} kan pas een visualisatie maken van de relaties tussen de bestanden als deze bestanden uitgebreid gedocumenteerd zijn. 
De relaties tussen de bestanden moeten expliciet in de documentatie vermeld zijn.
Doordat deze tool documentatie vergt, is er gekeken hoe deze relaties gegenereerd kunnen worden met behulp van LLM's.
Ook is er gekeken hoe Doxygen deze relaties visualiseert. Er wordt gebruik gemaakt van Graphviz van \textcite{GraphvizAuthors2024}.

Graphviz \autocite{GraphvizAuthors2024} kan echter niet gebruikt worden in een Python omgeving, de taal waarin dit onderzoek geschreven is.
Daarom is er gekeken naar een alternatief en soortgelijke tool om de relaties te tonen.
De tool Pyvis van \textcite{WHIR2018} is een Python library die te gebruiken is om graven te maken en te visualiseren.
Dit laat het toe om de relaties tussen de verschillende bestanden te visualiseren en een duidelijk overzicht te krijgen van het project.

\subsubsection{Genereren van de relaties tussen bestanden in een project}
\label{subsec:project-documentatie-relaties-genereren}

Om de relaties te bekomen tussen alle bestanden en mappen in een project is er een prompt meegegeven aan het Large Language Model.
In dit prompt worden de imports van alle bestanden opgelijst en wordt er gevraagd om de relaties tussen de bestanden weer te geven op basis van de prompts.
Een bestand dat een functie uit een ander bestand gebruikt, importeert deze functie impliciet. 
Daardoor wordt de functie die gebruikt wordt ook opgesomd onder de imports van het bestand.
Hierdoor kan er gekeken worden naar de imports van de verschillende bestanden en zo kunnen de relaties tussen de bestanden worden gevisualiseerd.

Het prompt dat gebruikt werd, is te zien in het codefragment~\ref{lst:generate-file-relations}.

\begin{listing}
    \begin{minted}{python}
'''
    You are an AI documentation assistant, and your task is to generate a csv file containing the relations between the files in a Python project.

    For the given project structure and imports:
    The imports are as follows:
    ...
    
    The structure of the project is as follows:
    ...
    
    The expected output of your task is the following:
    ```csv
File_Path,File_Name,Folder_Path,Uses_File
... 
```

    The Column "Uses File" should only contain the files where the file imports functions from.
    For example if the imports are:
    ```python
    ...
    from images import open_image
    from stages import stage1
    ```
    The Column "Uses File" should contain the file '4_Hangman_Game\\CMD_version\\images.py' and '4_Hangman_Game\\CMD_version\\stages.py'
    Do your task given the following imports and structure of the project:
    
    The imports are as follows:
    {imports}

    And the structure of the project is as follows:
    {structure}

    THE OUTPUT SHOULD BE A SINGLE CSV FILE CONTAINING THE RELATIONS BETWEEN THE FILES IN THE PROJECT.
    '''
\end{minted}
\caption{Prompt voor het genereren van de relaties tussen de bestanden in een project, vervoledigd in bijlage \ref{bijlage:generate-file-relations}}
\label{lst:generate-file-relations}
\end{listing}

Er zijn verschillende iteraties van dit prompt gemaakt om de beste resultaten te bekomen.
Deze iteraties en fouten in het prompt hebben enige tijd in beslag genomen om op te lossen, maar uiteindelijk zijn de kinderziektes opgelost.
Zo is het belangrijk dat er geen spaties in het voorbeeld CSV staan.
De uitkomst van dit prompt is een CSV bestand met daarin het pad van het bestand, de bestandsnaam, het pad van de folder waarin het bestand zit en een lijst van alle geimporteerde bestanden.
Omdat dit CSV bestand de basis is voor de graaf die later gemaakt wordt, is het belangrijk dat dit bestand correct gegenereerd wordt.
Ook zijn schrijffouten en onduidelijkheden in het prompt aangepast om een beter resultaat te bekomen.

Een verdere iteratie van het prompt laat het toe om alle imports in het CSV te laten staan. 
Er wordt achteraf gefilterd op de imports die niet relevant zijn. 
Wanneer er meerdere imports zijn, wordt er gevraagd aan de prompt om deze op te slaan in een lijst gescheiden door een puntkomma.

\subsubsection{Visualisatie van de relaties}
\label{subsec:project-documentatie-relaties-visualisatie}

Eens er een goed CSV bestand is bekomen, kunnen de relaties tussen de bestanden gevisualiseerd worden.
Dit wordt gedaan met behulp van de tool Pyvis \autocite{WHIR2018}.
Deze tool haalt de relaties uit het CSV bestand en voegt deze toe aan een graaf. 
Eerst worden de verschillende nodes toegevoegd en dan de edges tussen de nodes waar er een relatie is. 
De code waarop dit gebeurt is te vinden in bijlage \ref{bijlage:generate-file-graph}.
Als dit voor alle bestanden gedaan is, kan er een duidelijk overzicht bekomen worden van de relaties tussen de bestanden door de graaf te exporteren naar een HTML bestand.
Dit HTML bestand kan dan geopend worden in een browser om de graaf te bekijken, alsook kunnen de nodes versleept worden om een beter overzicht te bekomen.

\begin{figure}[h]
    \centering
    \includegraphics[width=0.5\textwidth]{graph.png}
    \caption{Voorbeeld van een graaf van de relaties tussen bestanden}
    \label{fig:graph}
\end{figure}

Er is een voorbeeld van een graaf te zien in de figuur \ref{fig:graph}. 
Voor een groot project is er geprobeerd of de gegenereerde graaf de relaties tussen de bestanden duidelijk kan weergegeven op een grotere schaal, een voorbeeld hiervan is te vinden in de figuur \ref{fig:graph-large}.
De relaties op deze graaf zijn zichtbaar, echter is het niet altijd duidelijk omdat er veel bestanden zijn en er bepaalde bestanden vaak geïmporteerd zijn.
Dit kan ervoor zorgen dat de graaf onoverzichtelijk wordt naargelang de grootte van het project.

\begin{figure}[h]
    \centering
    \includegraphics[width=1\textwidth]{graph-large.png}
    \caption{Voorbeeld van een fragment van een graaf van de relaties tussen bestanden van een groot project}
    \label{fig:graph-large}
\end{figure}

\section{Evaluatie}
\label{sec:project-documentatie-evaluatie}

\subsection{Inleiding}
\label{sec:project-documentatie-evaluatie-inleiding}

In dit hoofdstuk wordt er gekeken hoe de documentatie van een project geëvalueerd kan worden.
Dit wordt gedaan door de documentatie van de tool te vergelijken met de handgeschreven documentatie van een project.
Het bestand dat vergeleken wordt, is het Python-bestand met de naam AutoClicker van \textcite{Waegeneer2022} en het Python-project met de naam bmi-project van \textcite{Simmons2019}.

\subsection{Bestanddocumentatie evaluatie}
\label{sec:project-documentatie-evaluatie-bestand}

Door het testen van de moeilijkheidsgraden opgesteld in de tabel \ref{table:bestanden} kon de tool geëvalueerd worden.
De tool werkte zoals verwacht voor bestanden met moeilijkheidsgraad makkelijk tot moeilijk.
Het vergelijken van de zelfgedocumenteerde bestanden met de automatisch gegenereerde bestanden geeft een goed beeld van de kwaliteit van de gegenereerde docstrings en bestandsamenvattingen wat te zien is in de figuur \ref{fig:evaluatie-bestand-documentatie}.
Hieruit is te zien dat de functie omschrijvingen in de samenvatting anders verwoord zijn, maar de betekenis is hetzelfde.
Er is echter één functie vergeten in de manuele bestandsamenvatting, die wel gedocumenteerd is door de tool.
Een tweede fout in de automatisch gegenereerde bestandsamenvatting is dat de klasse \mintinline{python3}|MouseClick| een andere naam heeft gekregen namelijk \mintinline{python3}|AutoClicker| de naam van het bestand.
Dit is een goed resultaat, omdat de gegenereerde docstrings correct zijn en het geeft toch een duidelijk beeld van wat het bestand inhoudt.

\begin{figure}
    \centering
    \begin{subfigure}[b]{1\textwidth}
        \centering
        \includegraphics[width=1\textwidth]{zelf_autoclick.png}
        \caption{Zelfgedocumenteerde bestandsamenvatting. \ref{bijlage:zelfgedocumenteerd-bestand}}
        \label{fig:zelfgedocumenteerd-bestandsamenvatting}
    \end{subfigure}
    \hfill
    \begin{subfigure}[b]{1\textwidth}
        \centering
        \includegraphics[width=1\textwidth]{automatisch_autoclick.png}
        \caption{Automatisch gegenereerde bestandsamenvatting van eigen tool. \ref{bijlage:evaluatie-bestand-documentatie}}
        \label{fig:automatisch-bestandsamenvatting}
    \end{subfigure}
    \caption{Evaluatie van de automatisch gegenereerde bestandsamenvatting met de zelfgedocumenteerde bestandsamenvatting. Voor het bestand AutoClicker van \textcite{Waegeneer2022}}
    \label{fig:evaluatie-bestand-documentatie}
\end{figure}


\subsection{projectdocumentatie evaluatie}
\label{sec:project-documentatie-evaluatie-project}

De projectdocumentatie evaluatie bestaat uit het evalueren van de documentatie van een project gegenereerd door de tool en de handgeschreven documentatie van het project.
Dit wordt gedaan voor het project genaamd bmi-project van \textcite{Simmons2019}.

\begin{table}
    \resizebox{\textwidth}{!}{
    \begin{tabular}{|c|c|c|c|c|}
    \hline
    Type documentatie & Aantal die niet overeenkomen & Aantal exacte overeenkomsten & Aantal gelijkaardige overeenkomsten & Totale aantal \\
    \hline
    Docstring & 2 & 5 & 13 & 20 \\
    \hline
    Bestandsamenvatting & 3 & 2 & 3 & 8\\
    \hline
    \end{tabular}}
    \caption{Evaluatie van de documentatie van het project bmi-project van \textcite{Simmons2019}}
    \label{table:projecten-evaluatie}
\end{table}

Alle docstrings en bestandsamenvattingen zijn geëvalueerd en de resultaten zijn te zien in de tabel \ref{table:projecten-evaluatie}.
37.5\% van de automatisch gegenereerde samenvattingen hebben dezelfde betekenis als de handgeschreven samenvattingen. 25\% van de samenvattingen zijn exact hetzelfde.
Wanneer de bestandsamenvattingen niet overeenkomen is dit omdat er een klasse of functie extra toegevoegd is in de automatische gegenereerde samenvatting die niet in het orgineel bestand staat.
Een voorbeeld van deze foutieve bestandsamenvattingen is te zien in de figuur \ref{fig:evaluate-bestand-samenvatting}.

\begin{figure}
    \centering
    \begin{subfigure}[b]{0.5\textwidth}
        \centering
        \includegraphics[width=1\textwidth]{hand_bestanddocu.png}
        \caption{Zelfgedocumenteerde bestandsamenvatting.}
        \label{fig:zelfgedocumenteerd-bestandsamenvatting}
    \end{subfigure}
    \hfill
    \begin{subfigure}[b]{0.5\textwidth}
        \centering
        \includegraphics[width=1\textwidth]{auto_bestanddocu.png}
        \caption{Automatisch gegenereerde bestandsamenvatting van eigen tool.}
        \label{fig:automatisch-bestandsamenvatting}
    \end{subfigure}
    \caption{Vergelijking van de automatisch gegenereerde bestandsamenvatting met de zelfgedocumenteerde bestandsamenvatting.}
    \label{fig:evaluate-bestand-samenvatting}
\end{figure}

De docstrings komen voor 90\% overeen met de handgeschreven docstrings.
In 10\% van de andere gevallen is er een kleine fout in de docstring of is de docstring niet correct gegenereerd.
Dit gaat dan over een verkeerde omschrijving van de functie of klasse of een fout in de parameters.
Zo is er voor de docstring van de klasse \mintinline{python3}|MAECallback(Callback)| een fout in de automatisch gegenereerde docstring.
Het argument \mintinline{python3}|Callback| is niet meegegeven in de docstring, terwijl dit wel in de handgeschreven docstring staat.
Deze fout is te vinden in de figuur \ref{fig:evaluate-docstring}.

\begin{figure}
    \centering
    \begin{subfigure}[b]{0.5\textwidth}
        \centering
        \includegraphics[width=1\textwidth]{hand_docstring.png}
        \caption{Zelfgedocumenteerde docstring.}
        \label{fig:zelfgedocumenteerd-docstring}
    \end{subfigure}
    \hfill
    \begin{subfigure}[b]{0.5\textwidth}
        \centering
        \includegraphics[width=1\textwidth]{auto_docstring.png}
        \caption{Automatisch gegenereerde docstring van eigen tool.}
        \label{fig:automatisch-docstring}
    \end{subfigure}
    \caption{Vergelijking van de automatisch gegenereerde docstring met de zelfgedocumenteerde docstring.}
    \label{fig:evaluate-docstring}
\end{figure}

De handgeschreven projectsamenvatting heeft dezelfde boodschap als de automatisch gegenereerde samenvatting. Het is echter wel geschreven op een andere manier.
Doordat de samenvatting dezelfde boodschap overbrengt, kan er gesproken worden van een duidelijke samenvatting.
De oplijsting van de verschillende bestanden met elk de gedefinieerde functies en klassen is correct en duidelijk.

De evaluatie van de relaties tussen de verschillende bestanden gebeurt tussen de automatisch gegenereerde graaf en de manueel getekende graaf.
De graven, te zien in \ref{fig:evaluatie-graaf}, tonen de relaties tussen de bestanden van het project. 
Het is duidelijk dat deze graven gelijkaardig zijn. De gegenereerde graaf is beweegbaar en kan aangepast worden om een beter overzicht te krijgen.
Samen met de oplijsting van de bestanden en de graaf met de relaties tussen de bestanden kan er een duidelijk overzicht bekomen worden van het project.

\begin{figure}
    \centering
    \begin{subfigure}[b]{1\textwidth}
        \centering
        \includegraphics[width=0.5\textwidth]{graaf.png}
        \caption{Handgetekende graaf van de relaties tussen de bestanden van het project \autocite{Simmons2019}}
    \end{subfigure}
    \hfill
    \begin{subfigure}[b]{0.5\textwidth}
        \centering
        \includegraphics[width=1\textwidth]{generated_graaf.png}
        \caption{Gegenereerde graaf van de relaties tussen de bestanden van het project. \autocite{Simmons2019}}
    \end{subfigure}
    \caption{Vergelijking van de gegenereerde graaf met de handgetekende graaf.}
    \label{fig:evaluatie-graaf}
\end{figure}