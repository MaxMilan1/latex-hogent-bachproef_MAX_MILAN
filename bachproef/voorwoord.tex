%%=============================================================================
%% Voorwoord
%%=============================================================================

\chapter*{\IfLanguageName{dutch}{Woord vooraf}{Preface}}%
\label{ch:voorwoord}

%% TODO:
%% Het voorwoord is het enige deel van de bachelorproef waar je vanuit je
%% eigen standpunt (``ik-vorm'') mag schrijven. Je kan hier bv. motiveren
%% waarom jij het onderwerp wil bespreken.
%% Vergeet ook niet te bedanken wie je geholpen/gesteund/... heeft

Toen ik aan de richting Toegepaste Informatica begon, wist ik meteen welke specialisatie ik wilde volgen. 
De keuze richting AI en Data Engineering sprak mij direct aan. 
Deze richting biedt mij de mogelijkheid aan om steeds nieuwe uitdagingen te vinden in alles wat ik doe.

Het aangaan van een uitdaging geeft mij de motivatie om steeds het beste van mezelf te geven. 
Zo was het ook een uitdaging om deze bachelorproef tot een goed einde te brengen. 
Een van de grootste uitdagingen was het werken met LLM's, aangezien dit de eerste keer was dat ik deze zelf implementeerde.

Daarom wil ik graag mijn co-promotor, Arne Pannemans, bedanken voor zijn onophoudelijke steun en waardevolle bijdragen. 
Zijn voortdurende aanwezigheid bij het geven van feedback en zijn kritische blik waren van onschatbare waarde voor het tot stand brengen van deze bachelorproef. 
Zijn continue steun en toewijding hebben mij geïnspireerd en gemotiveerd om het beste uit mezelf te halen. 
Dankzij zijn begeleiding heb ik mijn onderzoek naar een hoger niveau kunnen tillen.

Ook wil ik mijn promotoren meneer Gert-Jan Bosteels en mevrouw Lena De Mol bedanken voor hun kritische en waardevolle feedback. 
Deze feedback heeft ervoor gezorgd dat ik de juiste richting uitging en dat ik mijn ideeën helder kon verwoorden.

Tot slot zou ik graag mijn ouders en vrienden willen bedanken. Hun steun, voortdurende aanmoediging en begrip hebben mij door alle moeilijke momenten geholpen. 
Dankzij hen heb ik deze bachelorproef tot een goed einde kunnen brengen.
