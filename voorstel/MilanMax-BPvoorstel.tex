%==============================================================================
% Sjabloon onderzoeksvoorstel bachproef
%==============================================================================
% Gebaseerd op document class `hogent-article'
% zie <https://github.com/HoGentTIN/latex-hogent-article>


\documentclass{hogent-article}

% Invoegen bibliografiebestand
\usepackage[backend=biber,style=apa]{biblatex}
\DeclareLanguageMapping{dutch}{dutch-apa}
\addbibresource{voorstel.bib}

% Informatie over de opleiding, het vak en soort opdracht
\studyprogramme{Professionele bachelor toegepaste informatica}
\course{Bachelorproef}
\assignmenttype{Onderzoeksvoorstel}

\academicyear{2023-2024}

% TODO: Werktitel
\title{Geautomatiseerde documentatie generatie met behulp van Large Language Modellen: Het genereren van duidelijke overzichten en informatieve beschrijvingen voor Python projecten}

% TODO: Studentnaam en emailadres invullen
\author{Max Milan}
\email{max.milan@student.hogent.be}

% TODO: Geef de co-promotor op
% \supervisor[Co-promotor]{S. Beekman (Synalco, \href{mailto:sigrid.beekman@synalco.be}{sigrid.beekman@synalco.be})}

% Binnen welke specialisatierichting uit 3TI situeert dit onderzoek zich?
% Kies uit deze lijst:
%
% - Mobile \& Enterprise development
% - AI \& Data Engineering
% - Functional \& Business Analysis
% - System \& Network Administrator
% - Mainframe Expert
% - Als het onderzoek niet past binnen een van deze domeinen specifieer je deze
%   zelf
%
\specialisation{AI \& Data Engineering}
\keywords{Documentatie, LLM, Large Language Modellen, Python, Project, Code Analyse}

\begin{document}

\begin{abstract}
Documentatie van een Python project is belangrijk, maar het is een tijdrovende taak en het wordt vaak niet grondig gedaan.
Deze bachelorproef aan de HoGent onderzoekt het automatisch genereren van documentatie voor python projecten met behulp van Large Language Modellen.
Er wordt een tool ontwikkeld die de Python code en de relaties tussen de verschillende bestanden analyseert en op basis daarvan een overzichtelijke documentatie genereerd.
Er wordt gekeken naar hoe de documentatie van Python functies gebruikt kunnen worden voor het maken van een gehele samenvatting van het project.
Dit wordt gedaan op basis van huidige methoden om docstrings aan te maken en te gebruiken.
Deze informatie kan dan gegeven worden aan de Large Language Modellen om een samenvatting te genereren.

Er worden verschillende Python projecten verzameld en geanalyseerd om te kijken hoe de documentatie gegenereerd kan worden.
Dan worden er LLMs getraind op basis van deze projecten en wordt er gekeken naar hoe de documentatie gegenereerd kan worden.
De gegenereerde documentatie kan dan vergeleken worden met de huidige documentatie van de projecten om dit te evalueren.
Ook zal er gevraagd worden aan enkele programmeurs om de documentatie te evalueren.

Op basis van deze feedback kan het model gefinetuned worden. Er kan gekeken worden naar de mogelijke verbeteringspunten zodat er uiteindelijk een betere documentatie van het project ontstaat.
Het resultaat is dat er een tool is die de documentatie van een Python project kan genereren.
Dit resultaat maakt het mogelijk om de gegenereerde samenvatting van een Python project te lezen. De lezer kan dan stukken gebruiken uit het project of er verder mee aan de slag gaan.
\end{abstract}

\tableofcontents

% De hoofdtekst van het voorstel zit in een apart bestand, zodat het makkelijk
% kan opgenomen worden in de bijlagen van de bachelorproef zelf.
%---------- Inleiding ---------------------------------------------------------

\section{Introductie}%
\label{sec:introductie}

Documentatie is belangrijk wanneer er aanpassingen moeten gebeuren aan de code van een project. Ook moet een iemand anders de code kunnen begrijpen voordat de code gebruikt kan worden binnen een ander project.
Hoe kan geautomatiseerde documentatiegeneratie met behulp van Large Language Modellen (LLM) effectief worden toegepast om een duidelijk overzichten en informatieve beschrijvingen te produceren voor Python projecten?

Het is belangrijk dat de skill of de knowhow van een project gedeeld kan worden met anderen. 
Door het toepassen van documentatie kan deze kennis makkelijk vergaard worden door andere geïnteresseerden.
Het is dus een belangrijk dat er aan documentatie gedaan wordt en dat deze up-to-date blijft. 

Documentatie is iets dat veel tijd kost, dat vaak niet gemaakt wordt en het is iets dat up-to-date gehouden moet worden.
Het gebruiken ervan kan ervoor zorgen dat er geen dubbel werk gedaan moet worden. Een tool die dit proces kan versnellen / automatiseren zou een grote meerwaarde zijn.
De tool bestaat uit een geautomatiseerde documentatie LLM die de project code analyseert en samenvat in een document. 
Dit geeft de werknemers de moegelijkheid om zich in te lezen in het project en erna zelf aanpassingen te maken of stukken code te gebruiken voor een ander project.

Het eindresultaat van deze bachelorproef is een Proof of Concept (PoC) van een geautomatiseerde tool die de project code analyseert en er documentatie van genereert.
De gegenereerde documentatie laat het toe om het project te begrijpen zonder er te veel tijd aan te besteden.

%---------- Stand van zaken ---------------------------------------------------

\section{Literatuurstudie}%
\label{sec:Literatuurstudie}

Wat is documentatie binnen Python projecten en wat zijn de huidige tools?
Voor de taal Python bestaan er al verschillende tools die documentatie genereren voor blokken code zoals pdoc \autocite{GallantHils2023} en Sphinx \autocite{Sphinx2023}. 
Met behulp van de sphinx autodoc functie \autocite{Sphinx2023} kan een python functie omschreven worden in een docstring.
Een docstring is een blok tekst dat de werking van een python functie omschrijft. Door deze beknopte blok tekst wordt er duidelijk wat de functie doet.
Deze docstrings kunnen mogelijks gebruikt worden bij het maken van een document dat het project omschrijft.

Er is ook al onderzoek gedaan naar het automatisch genereren van documentatie van code blokken met behulp van een Neural Attention Model (NAM) \autocite{IyerEtAl2016}.
Dit onderzoek heeft gekeken naar het genereren van hoogstaande samenvattingen van source code. 
Het maakt gebruik van neurale netwerken die stukken C\# code en SQL queries omzetten naar zinnen die de code omschrijven. 
Dit helpt bij het begrijpen van stukken code maar niet van een geheel project waar meerdere bestanden bij betrokken zijn.

Hoe kunnen Large Language Modellen gebruikt worden om documentatie te genereren?
Large Language Modellen zijn neurale netwerken die getraind worden op grote hoeveelheden tekst. 
Deze modellen kunnen tekst genereren op basis van een gegeven input. De mogelijkse invoer in deze bachelorproef kan dan een python project zijn, of de docstrings van verschillende functies in een project.
Er wordt dan verwacht dat het een samenvatting maakt van het project of van de functies.

Ook kan er gekeken worden naar GitHub README.md bestanden. Dit zijn bestanden waarin de werking van een project kort wordt uitgelegd. Deze zijn echter niet altijd makkelijk te lezen. 
Volgens de studie \textcite{GaoEtAl2023} kan de tekst vereenvoudigd worden terwijl steeds de correcte betekenis te behouden, en dit aan de hand van een transfer learning model.
Het gebruik van LLMs en het automatisch code documentatie met behulp van syntax bomen wordt onderzocht in \textcite{Procko2023} voor de C\# en .NET programeertalen.
In deze studie wordt er gekeken naar het gebruik van LLMs specifiek GPT-3.5 en GPT-4 om code te documenteren.

In de studie van \textcite{McBurneyMcMillan2014} wordt onderzocht hoe er automatisch documentatie gegenereerd kan worden voor Java code, specifiek naar hoe de methodes met elkaar verbonden zijn en welke rol ze spelen binnen het project.
Dit is een mooi voorbeeld van wat ik wil bereiken met deze bachelorproef, een samenhangend geheel van verschillende bestanden van een python project die samen een duidelijk overzicht geven van de werking van het project.
Hoe kunnen LLMs gebruikt worden om automatisch documentatie te genereren voor python projecten.
De gegenereerde samenvatting van het gehele project geeft de lezer ervan de mogelijkheid het project te gebruiken of aan te passen zonde de totale project code te ontleden.

%---------- Methodologie ------------------------------------------------------
\section{Methodologie}%
\label{sec:methodologie}

Hier beschrijf je hoe je van plan bent het onderzoek te voeren. Welke onderzoekstechniek ga je toepassen om elk van je onderzoeksvragen te beantwoorden? 
Gebruik je hiervoor literatuurstudie, interviews met belanghebbenden (bv.~voor requirements-analyse), experimenten, simulaties, vergelijkende studie, risico-analyse, PoC, \ldots?

Valt je onderwerp onder één van de typische soorten bachelorproeven die besproken zijn in de lessen Research Methods (bv.\ vergelijkende studie of risico-analyse)? 
Zorg er dan ook voor dat we duidelijk de verschillende stappen terug vinden die we verwachten in dit soort onderzoek!

Vermijd onderzoekstechnieken die geen objectieve, meetbare resultaten kunnen opleveren. 
Enquêtes, bijvoorbeeld, zijn voor een bachelorproef informatica meestal \textbf{niet geschikt}. 
De antwoorden zijn eerder meningen dan feiten en in de praktijk blijkt het ook bijzonder moeilijk om voldoende respondenten te vinden. 
Studenten die een enquête willen voeren, hebben meestal ook geen goede definitie van de populatie, 
waardoor ook niet kan aangetoond worden dat eventuele resultaten representatief zijn.

Uit dit onderdeel moet duidelijk naar voor komen dat je bachelorproef ook technisch voldoen\-de diepgang zal bevatten. 
Het zou niet kloppen als een bachelorproef informatica ook door bv.\ een student marketing zou kunnen uitgevoerd worden.

Je beschrijft ook al welke tools (hardware, software, diensten, \ldots) je denkt hiervoor te gebruiken of te ontwikkelen.

Probeer ook een tijdschatting te maken. Hoe lang zal je met elke fase van je onderzoek bezig zijn en wat zijn de concrete \emph{deliverables} in elke fase?

%---------- Verwachte resultaten ----------------------------------------------
\section{Verwacht resultaat, conclusie}%
\label{sec:verwachte_resultaten}

Hier beschrijf je welke resultaten je verwacht. Als je metingen en simulaties uitvoert, kan je hier al mock-ups maken van de grafieken samen met de verwachte conclusies. 
Benoem zeker al je assen en de onderdelen van de grafiek die je gaat gebruiken. 
Dit zorgt ervoor dat je concreet weet welk soort data je moet verzamelen en hoe je die moet meten.

Wat heeft de doelgroep van je onderzoek aan het resultaat? Op welke manier zorgt jouw bachelorproef voor een meerwaarde?

Hier beschrijf je wat je verwacht uit je onderzoek, met de motivatie waarom. Het is \textbf{niet} erg indien uit je onderzoek andere 
resultaten en conclusies vloeien dan dat je hier beschrijft: het is dan juist interessant om te 
onderzoeken waarom jouw hypothesen niet overeenkomen met de resultaten.



\printbibliography[heading=bibintec]

\end{document}